% $Id: qmmm.tex,v 1.3 1997-02-27 20:39:46 d3g681 Exp $

\label{sec:qmmm}

Combined or hybrid Quantum Mechanics and Molecular Mechanics (QM/MM)
is a simulation methodology that is about 15 years old but in all the
literature there are cautions that calibration computations must be
done to validate the model for each particular chemical system
studied.  This is not a black box style computation and the NWChem
users are advised that without calibration QM/MM may not give the
appropriate results\footnote{c.f., Singh and Kollman, J. Comp. Chem.
  {\bf 7}, 718 (1986); M.~J.~Field, P.~A.~Bash and M.~Karplus, J.
  Comp. Chem. {\bf 11}, 700, (1990); J. Gao, ``Methods and
  Applications of Combined Quantum Mechanical and Molecular Mechanical
  Potentials.'' In {\it Reviews in Computational Chemistry};
  K.~B.~Lipkowitz, D.~B.~Boyd, Eds.; VCH Publishers: New York, 199X;
  Vol. 7, pp 119-185 (1995); and M. A. Thompson and G. K. Schenter, J.
  Phys. Chem {\bf 99} 6374 (1995) }.

The QM/MM module in NWChem is driven by the molecular dynamics module
(nwArgos).  This module currently works for any QM method that has
analytic gradients.  The input for this requires the definition of
chemical system via the same interface that is used by the nwARGOS
module (c.f. Section \ref{sec:NWargos}).  The extensions to this
interface include the definition of ``Quantum'' atoms and ``Link''
where appropriate.  The QM information must be present in the
traditional NWChem input deck except for the geometry\footnote{Any
  geometry information in the traditional form will be ignored}.  The
geometrical information will be constructed automatically by nwARGOS.
For dynamics and free energy simulations the input is again identical
to that for nwARGOS with limitations on the kinds of simulations that
can be done.

The QM/MM module is invoked with the task directive where the
``theory'' is QMMM.  The recognized operations on the QM/MM theory
directive are energy, optimize, and dynamics.

\begin{verbatim}
  TASK QMMM (energy || optimize || dynamics)
\end{verbatim}

The compound input directive to set the scope of the QM/MM simulation
is: 
\begin{verbatim}
  QMMM
    ...
  END
\end{verbatim}

and \verb+QMMM+ includes the additional sub-directives that the user
may specify for the particular simulation.  These options are:

\section{QM method}
\begin{verbatim}
  QMETH <string (SCF || DFT || MP2 || MCSCF) default SCF>
\end{verbatim}





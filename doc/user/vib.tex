% $Id: vib.tex,v 1.9 1999-07-30 00:47:20 d3e129 Exp $
\label{sec:vib}

The nuclear hessian which is used to compute the vibrational
frequencies can be computed by finite difference for any ab initio
wave-function that has analytic gradients.  An analytic nuclear hessian
is available only for small, closed-shell SCF.  The appropriate
nuclear hessian generation algorithm is chosen based on the user input
when \verb+TASK <theory> frequencies+ is the task directive.

The vibrational package was integrated from the Utah Messkit and can
use any nuclear hessian generated from the driver routines.  There is
no required input for the ``VIB'' package.  VIB computes the
frequencies and intensities\footnote{Intensities are only computed if
  the dipole derivatives are available; these are computed by default
  for most methods that use the finite difference driver routines} for
for the computed nuclear hessian and the ``projected'' nuclear
hessian.  The VIB module projects out the translations and rotations
of the nuclear hessian using the standard Eckart projection algorithm.
The VIB module also computes the zero point energy for the molecular
system based on the frequencies obtained from the projected hessian.

The default mass of each atom is used unless an alternative mass is
provided via the geometry input, (c.f., \ref{sec:geom}).  The default
mass is the mass of the most abundant isotope of each
element.\footnote{c.f., "The Elements" by John Emsley, Oxford
  University Press, (C) 1989, ISBN 0-19-855237-8.} If the abundance
was roughly equal the mass of the isotope with the longest half life
was used.

\section{Animation} 
The ``VIB'' module also can generate mode animation input files in the
form standard xyz format files for graphics packages like
RasMol or XMol\footnote{There are scripts to automate this for RasMol in
.../nwchem/contrib/rasmolmovie}.  Each mode will have 20 xyz
files generated that cycle from the equilibrium geometry to 5 steps in
the positive direction of the mode vector, back to 5 steps in the
negative direction of the mode vector, and finally back to the
equilibrium geometry.  By default these files are {\bf not} generated.
\subsection{Activating the Animation File Generation}
To activate this mechanism simply use the following set directive 
\begin{verbatim}
set ``vib:animate'' logical true
\end{verbatim}
anywhere in the input deck prior to the task frequencies specification.  
\subsection{Controlling the Step Size Along the Mode Vector}
By default, the step size used is 0.15 a.u. which will give reliable
animation for most systems.  This can be changed via another set
directive:
\begin{verbatim}
set ``vib:animate:step_size'' real <step_size>
\end{verbatim}
where \verb+<step_size>+ is the real number that is the magnitude of
each step along the eigenvector of each nuclear hessian mode.

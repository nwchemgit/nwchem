% $Id: vib.tex,v 1.8 1998-03-10 23:25:37 d3g681 Exp $

The nuclear hessian which is used to compute the vibrational
frequencies can be computed by finite difference for any ab initio
wavefunction that has analytic gradients.  An analytic nuclear hessian
is available only for small, closed-shell SCF.  The appropriate
nuclear hessian generation algorithm is chosen based on the user input
when \verb+TASK <theory> frequencies+ is the task directive.

The vibrational package was integrated from the Utah Messkit and can
use any nuclear hessian generated from the driver routines.  There is
no required input for the ``VIB'' package.  VIB computes the
frequencies and intensities\footnote{Intensities are only computed if
  the dipole derivatives are available; these are computed by default
  for most methods that use the finite difference driver routines} for
for the comptued nuclear hessian and the ``projected'' nuclear
hessian.  The VIB module projects out the translations and rotations
of the nuclear hessian using the standard Eckart projection algorithm.
The VIB module also computes the zero point energy for the molecular
system based on the frequencies obtained from the projected hessian.

The default mass of each atom is used unless an alternative mass is
provided via the geometry input, (c.f., \ref{sec:geom}).  The default
mass is the mass of the most abundant isotope of each
element.\footnote{c.f., "The Elements" by John Emsley, Oxford
  University Press, (C) 1989, ISBN 0-19-855237-8.} If the abundance
was roughly equal the mass of the isotope with the longest half life
was used.



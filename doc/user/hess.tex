% $Id$
\label{sec:hess}

This section relates to the computation of analytic hessians which
are available for open and closed shell SCF, except ROHF and for closed shell
DFT.  Analytic hessians are not currently available for SCF or DFT
calculations relativistic all-electron methodologies or for charge fitting with DFT.  The current
algorithm is fully in-core and does not use symmetry.  This will be
changed in the next release.

There is no required input for the Hessian module.  This module only
impacts the hessian calculation.  For options for calculating the
frequencies, please see Section \ref{sec:vib}, the Vibrational module.

\section{Hessian Module Input}

All input for the Hessian Module is optional since the default
definitions are usually correct for most purposes.  
The generic module input begins with \verb+hessian+
and has the form:
\begin{verbatim}
  hessian
    thresh <real tol default 1d-6>
    print ...
    profile
  end
\end{verbatim}

\subsection{Defining the wavefunction threshold}
You may modify the default threshold for the wavefunction.  This keyword
is identical to THRESH in the SCF, Section \ref{sec:thresh}, and the
CONVERGENCE gradient in the DFT, Section \ref{sec:dftconv}.  The usual
defaults for the convergence of the wavefunction for single point and
gradient calculations is generally not tight enough for analytic hessians.
Therefore, the hessian, by default, tightens these up to 1d-6 and runs
an additional energy point if needed.  If,
during an analytic hessian calculation, you encounter an error:
\begin{verbatim}
cphf_solve:the available MOs do not satisfy the SCF equations
\end{verbatim}
the convergence criteria of the wavefunction generally needs to be
tightened.

\subsection{Profile}
The PROFILE keyword provides additional information concerning the 
computation times of
different sections of the hessian code.  Summary information is
given about the maximum, minimum and average times that a particular
section of the code took to complete.  This is normally only useful
for developers.

\subsection{Print Control} 
Known controllable print options are:

\begin{table}[h]
\begin{center}
\begin{tabular}{lcc}
  {\bf Name}          & {\bf Print Level} & {\bf Description} \\
 ``hess\_follow''               & high        & more information about where the calculation is \\
 ``cphf\_cont''                 & debug       & detailed CPHF information \\
 ``nucdd\_cont''                & debug       & detailed nuclear contribution information \\
 ``onedd\_cont''                & debug       & detailed one electron contribution information \\
 ``twodd\_cont''                & debug       & detailed two electron contribution information \\
 ``fock\_xc''                   & debug       & detailed XC information during the fock builds \\
\end{tabular}
\end{center}
\caption{Hessian Print Control Specifications}
\end{table}





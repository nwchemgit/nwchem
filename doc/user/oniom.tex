%
% $Id$
%
\label{sec:oniom}

ONIOM is the hybrid method of Morokuma and co-workers that enables
different levels of theory to be applied to different parts of a
molecule/system and combined to produce a consistent energy
expression.  The objective is to perform a high-level calculation on
just a small part of the system and to include the effects of the
remainder at lower levels of theory, with the end result being of
similar accuracy to a high-level calculation on the full system.

\begin{enumerate}
\item M. Svensson, S. Humbel, R.D.J. Froese, T. Mastubara, S. Sieber, and
K. Morokuma, J.~Phys.~Chem., 100, 19357 (1996).
\item  S. Dapprich, I. Komaromi, K.S. Byun, K. Morokuma, and M.J. Frisch,
J.~Mol.~Struct.~(Theochem), 461-462, 1 (1999).
\item R.D.J. Froese and K. Morokuma in ``Encylopedia of Computational Chemistry,'' 
volume 2, pp.1244-1257, (ed. P. von Rague Schleyer, John Wiley and Sons, 
Chichester, Sussex, 1998).
\end{enumerate}

The NWChem ONIOM module implements two- and three-layer ONIOM models
for use in energy, gradient, geometry optimization, and vibrational
frequency calculations with any of the pure quantum mechanical methods
within NWChem.  At the present time, it is not possible to perform
ONIOM calculations with either solvation models or classical force
fields.  Nor is it yet possible to compute properties except as
derivatives of the total energy.  

Using the terminology of Morokuma et al., the full molecular geometry
including all atoms is referred to as the ``real'' geometry and it is
treated using a ``low''-level of theory.  A subset of these atoms
(referred to as the ``model'' geometry) are treated using both the
``low''-level and a ``high''-level of theory.  A three-layer model
also introduces an ``intermediate'' model geometry and a ``medium''
level of theory.

The two-layer model requires a high and low level of theory and a
real and model molecular geometry.  The energy at the high-level of
theory for the real geometry is estimated as
\begin{verbatim}
      E(High,Real) = E(Low,Real) + [E(High,Model) - E(Low,Model)].
\end{verbatim}
The three-layer model requires high, medium and low levels of theory,
and real, intermediate and model geometries and the corresponding
energy estimate is
\begin{verbatim}
      E(High,Real) = E(Low,Real) + [E(High,Model) - E(Medium,Model)]
                     +  [E(Medium,Inter) - E(Low,Inter)].
\end{verbatim}

When does ONIOM work well?  The approximation for a two-layer model
will be good if
\begin{itemize}
\item the model system includes the interactions that dominate the
   energy difference being computed and the high-level of theory
   describes these to the required precision, and
\item the interactions between the model and the rest of the real system
    (substitution effects) are described to sufficient accuracy at the
    lower level of theory.
\end{itemize}
ONIOM is used to compute energy differences and the absolute energies
are not all that meaningful even though they are well defined.  Due to
cancellation of errors, ONIOM actually works better than you might
expect, but a poorly designed calculation can yield very bad results.
Please read and heed the caution at the end of the article by Dapprich
et al.

The input options are as follows
\begin{verbatim}
ONIOM
  HIGH   <string theory> [basis <string basis default "ao basis">] \
                         [ecp <string ecp>] [input <string input>]
 [MEDIUM <string theory> [basis <string basis default "ao basis">] \
                         [ecp <string ecp>] [input <string input>]]
  LOW    <string theory> [basis <string basis default "ao basis">] \
                         [ecp <string ecp>] [input <string input>]
  MODEL <integer natoms> [charge <double charge>] \
                         [<integer i1 j1> <real g1> [<string tag1>] ...]
 [INTER <integer natoms> [charge <double charge>] \
                         [<integer i1 j1> <real g1> [<string tag1>] ...]]
 [VECTORS [low-real <string mofile>] [low-model <string mofile>] \
          [high-model <string mofile>] [medium-model <string mofile]\
          [medium-inter <string mofile>] [low-inter <string mofile>]]
 [PRINT ...]
 [NOPRINT ...]
END
\end{verbatim}
which are described in detail below.  

{\em  For better validation of user input, the \verb+HIGH+,
\verb+LOW+ and \verb+MODEL+ directives must always be specified.  If
the one of the \verb+MEDIUM+ or \verb+INTER+ directives are specified,
then so must the other.}

\section{Real, model and intermediate geometries}

The geometry and total charge of the full or real system should be
specified as normal using the geometry directive (see Section
\ref{sec:geom}).  If $N_{model}$ of the atoms are to be included in
the model system, then these should be specified first in the
geometry.  Similarly, in a three-layer calculation, if there are
$N_{inter}$ atoms to be included in the intermediate system, then
these should also be arranged together at the beginning of the
geometry.  The implict assumption is that the model system is a subset
of the intermediate system which is a subset of the real system.  The
number of atoms to be included in the model and intemediate systems
are specified using the \verb+MODEL+ and \verb+INTER+ directives.
Optionally, the total charge of the model and intermediate systems may
be adjusted.  The default is that all three systems have the same
total charge.

Example 1.  A two-layer calculation on $K^{+}(H_2O)$ taking the
potassium ion as the model system.  Note that no bonds are broken so
no link atoms are introduced.  The real geometry would be specified
with potassium (the model) first.
\begin{verbatim}
    geometry autosym
      K  0    0.00    1.37
      O  0    0.00   -1.07
      H  0   -0.76   -1.68
      H  0    0.76   -1.68
    end
\end{verbatim}
and the following directive in the ONIOM input block indicates that
one atom (implicitly the first in the geometry) is in the model system
\begin{verbatim}
    model 1
\end{verbatim}

\subsection{Link atoms}
Link atoms for bonds spanning two regions are automatically generated
from the bond information.  The additional parameters on the
\verb+MODEL+ and \verb+INTER+ directives describe the broken bonds
including scale factors for placement of the link atom 
and, optionally, the type of link atom.  The type of link atom
defaults to hydrogen, but any type may be specified (actually here you
are specifying a geometry tag which is used to associate a geometrical
center with an atom type and basis sets, etc..  See section 
\ref{sec:cart}).
For each broken bond specify the numbers of the two atoms (i and j),
the scale factor (g) and optionally the tag of the link atom.  Link
atoms are placed along the vector connecting the the first to the
second atom of the bond according to the equation
\begin{displaymath}
\underline{R}_{link} = (1-g)\underline{R}_{1} + g*\underline{R}_{2}
\end{displaymath}
where $g$ is the scale factor.  If the scale factor is one, then the
link atom is placed where the second atom was.  More usually, the
scale factor is less than one, in which case the link atom is placed
between the original two atoms.  The scale factor should be chosen so
that the link atom (usually hydrogen) is placed near its equilibrium
bond length from the model atom.  E.g., when breaking a single
carbon-carbon bond (typical length 1.528 {\angstroms}) using a hydrogen
link atom we will want a carbon-hydrogen bond length of about 1.084
{\angstroms}, so the scale factor should be chosen as $1.084/1.528
\approx 0.709$.

Example 2.  A calculation on acetaldehyde ($H_3C-CHO$) using aldehyde
($H-CHO$) as the model system. The covalent bond between the two
carbon atoms is broken and a link atom must be introduced to replace
the methyl group.  The link atom is automatically generated --- all
you need to do is specify the atoms in the model system that are also
in the real system (here $CHO$) and the broken bonds.  Here is the
geometry of acetaldehyde with the $CHO$ of aldehyde first
\begin{verbatim}
    geometry
      C    -0.383    0.288    0.021
      H    -1.425    0.381    0.376
      O     0.259    1.263   -0.321

      H     0.115   -1.570    1.007
      H    -0.465   -1.768   -0.642
      H     1.176   -1.171   -0.352
      C     0.152   -1.150    0.005
    end
\end{verbatim}
There are three atoms (the first three) of the real geometry included
in the model geometry, and we are breaking the bond between atoms 1
and 7, replacing atom 7 with a hydrogen link atom.  This is all
accomplished by the directive
\begin{verbatim}
    model 3   1 7 0.709 H
\end{verbatim}
Since the default link atom is hydrogen there is actually no need to
specify the ``H''.

See also Section \ref{sec:oniomeg3} for a more complex example.

\subsection{Numbering of the link atoms}

The link atoms are appended to the atoms of the model or intermediate
systems in the order that the broken bonds are specified in the input.
This is of importance only if manually constructing an initial guess.

\section{High, medium and low theories}

The two-layer model requires both the high-level and low-level
theories be specified.  The three-layer model also requires the
medium-level theory.  Each of these includes a theory (such as SCF,
MP2, DFT, CCSD, CCSD(T), etc.), an optional basis set, an optional ECP,
and an optional string containing general NWChem input.

\subsection{Basis specification}
The basis name on the theory directive (high, medium, or low) is that
specified on a basis set directive (see Section \ref{sec:basis}) and
{\em not} the name of a standard basis in the library.  If not
specified, the basis set for the high-level theory defaults to the
standard \verb+"ao basis"+.  That for the medium level defaults to the
high-level basis, and the low-level basis defaults to the medium-level
basis.  Other wavefunction parameters are obtained from the standard
wavefunction input blocks.  See \ref{sec:oniomeg2} for an example.

\subsection{Effective core potentials}

If an effective core potential is specified in the usual fashion (see
Section \ref{sec:ecp}) outside of the ONIOM input then this will be
used in all calculations.  If an alternative ECP name (the name
specified on the ECP directive in the same manner as done for basis
sets) is specified on one of the theory directives, then this ECP will
be used in preference for that level of theory.  See Section
\ref{sec:oniomeg2} for sample input.

\subsection{General input strings}

For many purposes, the ability to specify the theory, basis and
effective core potential is adequate.  All of the options for each
theory are determined from their independent input blocks.  However,
if the same theory (e.g., DFT) is to be used with different options
for the ONIOM theoretical models, then the general input strings must
be used.  These strings are processed as NWChem input each time the
theoretical model is invoked.  The strings may contain any NWChem
input, except for options pertaining to ONIOM and the task directive.
The intent that the strings be used just to control the options
pertaining to the theory being used.

A word of caution.  Be sure to check that the options are producing
the desired results.  Since the NWChem database is persistent and the
ONIOM calculations happen in an undefined order, the input strings
should fully define the calculation you wish to have happen.

For instance, if the high model is DFT/B3LYP/6-311g** and the
low model is DFT/LDA/3-21g, the ONIOM input might look like this
\begin{verbatim}
    oniom
      model 3
      low  dft basis 3-21g    input "dft\; xc\; end"
      high dft basis 6-311g** input "dft\; xc b3lyp\; end"
    end
\end{verbatim}
The empty \verb+XC+ directive restores the default LDA
exchange-correlation option (see Section \ref{sec:xc}).  Note that
semi-colons and other quotation marks inside the input string must be
preceded by a backslash to avoid special interpretation.

See Section \ref{sec:oniomeg4} for another example.

\section{Use of symmetry}

Symmetry should work just fine as long as the model and intermediate
regions respect the symmetry --- i.e., symmetry equivalent atoms need
to be treated equivalently.  If symmetry equivalent atoms must be
treated in separate regions then the symmetry must be lowered (or
completely switched off).  

\section{Molecular orbital files}

The \verb+VECTORS+ directive in the ONIOM block is different to that
elsewhere in NWChem.  For each of the necessary combinations of theory
and geometry you can specify a different file for the molecular
orbitals.  By default each combination will store the MO vectors in
the permanent directory using a file name created by appending to the
name of the calculation the following string
\begin{itemize}
\item low-real  --- \verb+".lrmos"+
\item low-inter --- \verb+".limos"+
\item low-model --- \verb+".lmmos"+
\item medium-inter --- \verb+".mimos"+
\item medium-model --- \verb+".mmmos"+
\item high-model --- \verb+".hmmos"+
\end{itemize}
Each calculation will utilize the appropriate vectors which is more
efficient during geometry optimizations and frequency calculations,
and is also useful for the initial calculation.  In the absence of
existing MO vectors files, the default atomic guess is used (see
Section \ref{sec:vectors}).

If special measures must be taken to converge the initial SCF, DFT or
MCSCF calculation for one or more of the systems, then initial vectors
may be saved in a file with the default name, or another name may be
specified using the \verb+VECTORS+ directive.  Note that subsequent
vectors (e.g., from a geometry optimization) will be written back to
this file, so take a copy if you wish to preserve it.  
To generate the initial guess for the model or intermediate systems
it is necessary to generate the geometries which is most readily
done, if there are link atoms, by just running NWChem on the
input for the ONIOM calculation on your workstation.  It will
print these geometries before starting any calculations which 
you can then terminate.

E.g., in a calculation on Fe(III) surrounded by some ligands, it is
hard to converge the full (real) system from the atomic guess so as to
obtain a $d^5$ configuration for the iron atom since the $d$ orbitals
are often nominally lower in energy than some of the ligand orbitals.
The most effective mechanism is to converge the isolated Fe(III) and
then to use the fragment guess (see Section \ref{sec:fragguess}) as a
starting guess for the real system.  The resulting converged molecular
orbitals can be saved either with the default name (as described above
in this section), in which case no additional input is necessary.  If
an alternative name is desired, then the \verb+VECTORS+ directive may
be used as follows
\begin{verbatim}
    vectors low-real /u/rjh/jobs/fe_ether_water.mos
\end{verbatim}

\section{Restarting}

Restart of ONIOM calculations does not currently work as smoothly as
we would like.  For geometry optimizations that terminated gracefully
by running out of iterations, the restart will work as normal.
Otherwise, specify in the input of the restart job the last geometry
of the optimization.  The Hessian information will be reused and the
calculation should proceed losing at most the cost of one ONIOM
gradient evaluation.  For energy or frequency calculations, restart
may not currently be possible.

\section{Examples}

\subsection{Hydrocarbon bond energy}
\label{sec:oniomeg1}

A simple two-layer model changing just the wavefunction with one
link atom.

This reproduces the two-layer ONIOM (MP2:HF) result from Dapprich et
al.\ for the reaction $R-CH_3 = R-CH_2 + H$ with $R=CH_3$ using $CH_4$
as the model .  The geometries of $R-CH_3$ and $R-CH_2$ are optimized
at the DFT-B3LYP/6-311++G** level of theory, and then ONIOM is used to
compute the binding energy using UMP2 for the model system and HF for
the real system.  The results, including MP2 calculations on the full
system for comparison, are as given in Table \ref{tab:oniom1}

\begin{table}[h]
\begin{center}
\begin{tabular}{lccccc}
 Theory &   Me-CH2   &   Me-Me   &   H       & De(Hartree)&  De(kcal/mol) \\ \hline
 B3LYP  &  -79.185062& -79.856575&  -0.502256&  0.169257 &   106.2 \\
 HF     &  -78.620141& -79.251701&  -0.499817&  0.131741 &    82.7 \\
 MP2    &  -78.904716& -79.571654&  -0.499817&  0.167120 &   104.9 \\
 MP2:HF &  -78.755223& -79.422559&  -0.499817&  0.167518 &   105.1 \\ \hline
\end{tabular}
\caption{\label{tab:oniom1} Energies for ONIOM example 1, hydrocarbon bond energy using MP2:HF two-layer model.}
\end{center}
\end{table}

The following input first performs a calculation on $CH_3-CH_2$, and then
on $CH_3-CH_3$.  Note that in the second calculation we cannot use the
full symmetry since we are breaking the C-C bond in forming the model
system (the non-equivalence of the methyl groups is perhaps more
apparent if we write $R-CH_3$).

\begin{verbatim}
    start

    basis spherical
      H library 6-311++G**; C library 6-311++G**
    end

    title "ONIOM Me-CH2"

    geometry autosym
      H    -0.23429328     1.32498565     0.92634814
      H    -0.23429328     1.32498565    -0.92634814
      C    -0.13064265     0.77330370     0.00000000
      H    -1.01618703    -1.19260361     0.00000000
      H     0.49856072    -1.08196901    -0.88665533
      H     0.49856072    -1.08196901     0.88665533
      C    -0.02434414    -0.71063687     0.00000000
    end

    scf; uhf; doublet; thresh 1e-6; end
    mp2; freeze atomic; end

    oniom
      high mp2
      low  scf
      model 3   3 7 0.724
    end

    task oniom

    title "ONIOM Me-Me"

    geometry   # Note cannot use full D3D symmetry here
      H   -0.72023641     0.72023641    -1.16373235
      H    0.98386124     0.26362482    -1.16373235
      H   -0.26362482    -0.98386124    -1.16373235
      C    0.00000000     0.00000000    -0.76537515
      H    0.72023641    -0.72023641     1.16373235
      H   -0.98386124    -0.26362482     1.16373235
      H    0.26362482     0.98386124     1.16373235
      C    0.00000000     0.00000000     0.76537515
    end

    scf; rhf; singlet; end

    oniom
      high mp2
      low  scf
      model 4   4 8 0.724
    end

    task oniom
\end{verbatim}

\subsection{Optimization and frequencies}
\label{sec:oniomeg2}
A two-layer model including modification of theory, basis, ECP and
total charge and no link atoms.

This input reproduces the ONIOM optimization and vibrational frequency
calculation of $Rh(CO)_2Cp$ of Dapprich et al.  The model system is
$Rh(CO)_2^+$.  The low theory is the Gaussian LANL2MB model (Hay-Wadt
n+1 ECP with minimal basis on Rh, STO-3G on others) with SCF.  The
high theory is the Gaussian LANL2DZ model (another Hay-Wadt ECP with a
DZ basis set on Rh, Dunning split valence on the other atoms) with
DFT/B3LYP.  Note that different names should be used for the basis set
and ECP since the same mechanism is used to store them in the
database.

\begin{verbatim}
    start

    ecp LANL2DZ_ECP
      rh library LANL2DZ_ECP
    end

    basis LANL2DZ spherical
      rh library LANL2DZ_ECP
      o library SV_(Dunning-Hay); c library SV_(Dunning-Hay); h library SV_(Dunning-Hay)
    end

    ecp Hay-Wadt_MB_(n+1)_ECP
      rh library Hay-Wadt_MB_(n+1)_ECP
    end

    # This is the minimal basis used by Gaussian.  It is not the same 
    # as the one in the EMSL basis set library for this ECP.
    basis Hay-Wadt_MB_(n+1) spherical 
      Rh s; .264600D+01 -.135541D+01; .175100D+01  .161122D+01; .571300D+00  .589381D+00
      Rh s; .264600D+01  .456934D+00; .175100D+01 -.595199D+00; .571300D+00 -.342127D+00
            .143800D+00  .410138D+00; .428000D-01  .780486D+00
      Rh p; .544000D+01 -.987699D-01; .132900D+01  .743359D+00; .484500D+00  .366846D+00
      Rh p; .659500D+00 -.370046D-01; .869000D-01  .452364D+00; .257000D-01  .653822D+00
      Rh d; .366900D+01  .670480D-01; .142300D+01  .455084D+00; .509100D+00  .479584D+00
            .161000D+00  .233826D+00
      o  library sto-3g; c  library sto-3g; h  library sto-3g
    end

    charge 0
    geometry autosym
      rh       0.00445705    -0.15119674     0.00000000
      c       -0.01380554    -1.45254070     1.35171818
      c       -0.01380554    -1.45254070    -1.35171818
      o       -0.01805883    -2.26420212     2.20818932
      o       -0.01805883    -2.26420212    -2.20818932
      c        1.23209566     1.89314720     0.00000000
      c        0.37739392     1.84262319    -1.15286640
      c       -1.01479160     1.93086461    -0.70666350
      c       -1.01479160     1.93086461     0.70666350
      c        0.37739392     1.84262319     1.15286640
      h        2.31251453     1.89903673     0.00000000
      h        0.70378132     1.86131979    -2.18414218
      h       -1.88154273     1.96919306    -1.35203550
      h       -1.88154273     1.96919306     1.35203550
      h        0.70378132     1.86131979     2.18414218
    end

    dft; grid fine; convergence gradient 1e-6 density 1e-6; xc b3lyp; end
    scf; thresh 1e-6; end

    oniom
      low scf basis Hay-Wadt_MB_(n+1) ecp  Hay-Wadt_MB_(n+1)_ECP
      high dft basis LANL2DZ ecp LANL2DZ_ECP
      model 5 charge 1
      print low
    end

    task oniom optimize
    task oniom freq
\end{verbatim}

\subsection{A three-layer example}
\label{sec:oniomeg3}

A three layer example combining CCSD(T), and MP2 with two different
quality basis sets, and using multiple link atoms.

The full system is tetra-dimethyl-amino-ethylene (TAME) or
(N(Me)2)2-C=C-(N(Me)2)2.  The intermediate system is (NH2)2-C=C-(NH2)2
and H2C=CH2 is the model system.  CCSD(T)+aug-cc-pvtz is used for the
model region, MP2+aug-cc-pvtz for the intermediate region, and
MP2+aug-cc-pvdz for everything.  

In the real geometry the first two atoms (C, C) are the model system
(link atoms will be added automatically).  The first six atoms (C, C,
N, N, N, N) describe the intermediate system (again with link atoms to
be added automatically).  The atoms have been numbered using comments
to make the bonding input easier to generate.  

To make the model system, four C-N bonds are broken between the
ethylene fragment and the dimethyl-amino groups and replaced with C-H
bonds.  To make the intermediate system, eight C-N bonds are broken
between the nitrogens and the methyl groups and replaced with N-H
bonds.  The scaling factor could be chosen differently for each of the
bonds.

\begin{verbatim}
    start 

    geometry 
      C  0.40337795 -0.17516305 -0.51505208  # 1
      C -0.40328664  0.17555927  0.51466084  # 2
      N  1.87154979 -0.17516305 -0.51505208  # 3
      N -0.18694782 -0.60488524 -1.79258692  # 4
      N  0.18692927  0.60488318  1.79247594  # 5
      N -1.87148219  0.17564718  0.51496494  # 6
      C  2.46636552  1.18039452 -0.51505208  # 7
      C  2.48067731 -1.10425355  0.46161675  # 8
      C -2.46642715 -1.17982091  0.51473105  # 9
      C -2.48054940  1.10495864 -0.46156202  # 10
      C  0.30027136  0.14582197 -2.97072148  # 11
      C -0.14245927 -2.07576980 -1.96730852  # 12
      C -0.29948109 -0.14689874  2.97021079  # 13
      C  0.14140463  2.07558249  1.96815181  # 14
      H  0.78955302  2.52533887  1.19760764
      H -0.86543435  2.50958894  1.88075113
      ... and 22 other hydrogen atoms on the methyl groups
    end

    basis aug-cc-pvtz spherical
      C library aug-cc-pvtz; H library aug-cc-pvtz
    end

    basis aug-cc-pvdz spherical
      C library aug-cc-pvtz; H library aug-cc-pvtz
    end

    oniom      
      high ccsd(t) basis aug-cc-pvtz
      medium mp2 basis aug-cc-pvtz
      low mp2 basis aug-cc-pvdz
      model 2   1 3  0.87   1 4  0.87   2 5  0.87   2 6  0.87

      inter 6   3 7  0.69   3 8  0.69   4 11 0.69   4 12 0.69 \
                5 13 0.69   5 14 0.69   6 9  0.69   6 10 0.69
    end

    task oniom
\end{verbatim}

\subsection{DFT with and without charge fitting}
\label{sec:oniomeg4}
Demonstrates use of general input strings.

A two-layer model for anthracene (a linear chain of three fused benzene
rings) using benzene as the model system.  The high-level theory is
DFT/B3LYP/TZVP with exact Coulomb. The low level is DFT/LDA/DZVP2 with
charge fitting.  

Note the following.  
\begin{enumerate}
\item The semi-colons and quotation marks inside the input string must be 
quoted with backslash.
\item The low level of theory sets the fitting basis set and the high level of 
theory unsets it.
\end{enumerate}

\begin{verbatim}
    start
    geometry 
      symmetry d2h
      C    0.71237329    -1.21458940     0.0
      C   -0.71237329    -1.21458940     0.0
      C    0.71237329     1.21458940     0.0
      C   -0.71237329     1.21458940     0.0
      C   -1.39414269     0.00000000     0.0
      C    1.39414269     0.00000000     0.0
      H   -2.47680865     0.00000000     0.0
      H    2.47680865     0.00000000     0.0
      C    1.40340535    -2.48997027     0.0
      C   -1.40340535    -2.48997027     0.0
      C    1.40340535     2.48997027     0.0
      C   -1.40340535     2.48997027     0.0
      C    0.72211503     3.64518615     0.0
      C   -0.72211503     3.64518615     0.0
      C    0.72211503    -3.64518615     0.0
      C   -0.72211503    -3.64518615     0.0
      H    2.48612947     2.48094825     0.0
      H    1.24157357     4.59507342     0.0
      H   -1.24157357     4.59507342     0.0
      H   -2.48612947     2.48094825     0.0
      H    2.48612947    -2.48094825     0.0
      H    1.24157357    -4.59507342     0.0
      H   -1.24157357    -4.59507342     0.0
      H   -2.48612947    -2.48094825     0.0
    end

    basis small
      h library DZVP_(DFT_Orbital)
      c library DZVP_(DFT_Orbital)
    end

    basis fitting
      h library DGauss_A1_DFT_Coulomb_Fitting
      c library DGauss_A1_DFT_Coulomb_Fitting
    end

    basis big
      h library TZVP_(DFT_Orbital)
      c library TZVP_(DFT_Orbital)
    end

    oniom
      model 8   1 9 0.75   2 10 0.75   3 11 0.75   4 12 0.75
      high dft basis big   input "unset \"cd basis\"\; dft\; xc b3lyp\; end"
      low  dft basis small input "set \"cd basis\" fitting\; dft\; xc\; end"
    end

    task oniom
\end{verbatim}

\begin{verbatim}
  GRADIENTS 
    [CHKPT <integer minutes>]
    [RESTART]
    [PRINT]
    [NOPRINT]
  END
\end{verbatim}

  This input controls the Hartree-Fock (SCF, UHF and ROHF) gradients.  

\subsection{CHKPT}

  This keyword is used to specify a time interval after which a
  checkpoint for later restart is created. After \verb+minutes+
  minutes of walltime the forces are written to the runtime database.

\subsection{RESTART}

  This keyword tells the program that this is a restart of an aborted
  gradient calculation. The partially calculated forces are taken from
  the database of the previous run. If they are not present, the
  keyword is ignored and a complete calculation of the gradients is started.

  It also works within a geometry optimization. Subsequent gradient 
  calculations are not treated as restarts.

\subsection{PRINT, NOPRINT}

  Currently only some print control is available.

\begin{tabbing}
  Very\_long\_descriptive\_name \= Print level space \= \kill
  Name                   \> Print Level \> Description \\
                         \>        \> \\
        'information'   \>        low  \> calculation info\\
        'geometry'    \>          high \> \\
        'basis'        \>         high \> \\
        'forces'   \>             low \> \\
        'timing'   \>             default \> 
\end{tabbing}

\subsection{frozen atoms}
\label{sec:activeatoms}

Currently the only mechanism for freezing atoms is to enter a list of
active atoms via the \verb+set+ directive.

\begin{verbatim}
  set geometry:actlist integer <at1> <at2> <at3> ...
\end{verbatim}
defines atoms number \verb+<at1>+ \ldots as 'active', and only forces
on those are calculated. All other atoms remain frozen at their
starting coordinates during a geometry optimization.
% I have no idea how this works with symmetry.
% But in the new release there will be frozen atoms and variables
% anyway.


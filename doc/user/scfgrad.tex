%
% $Id$
%
\label{sec:scfgrad}


The input for this directive allows the user to adjust the print control
for the SCF, UHF, ROHF, MCSCF and MP2 gradients.  The
form of the directive is as follows:

% RJH removed 
%    [chkpt <integer minutes>]
%    [restart]

\begin{verbatim}
  GRADIENTS 
    [print || noprint] ...
  END
\end{verbatim}

%The directive contains two keywords, \verb+chkpt+ and \verb+restart+,
%that are related to the creation of the gradients.  The keyword
%\verb+chkpt+ allows the user to specify a time interval at which the
%current values for the forces that make up the gradient are saved, for
%access by a later calculation.  The time interval is specified in the
%integer variable \verb+minutes+, and defines the number of minutes of
%elapsed wall-clock time since the start of the calculation when the
%gradient is written to the runtime database.
%RJH: No default chkpt value?  What happens if this is not specified?--fmr.

% \section{CHKPT}

%  This keyword is used to specify a time interval after which a
%  checkpoint for later restart is created. After \verb+minutes+
%  minutes of walltime the forces are written to the runtime database.

% \section{RESTART}

%Specifying the keyword \verb+restart+ allows the user to restart a calculation
%using the gradient from a previous calculation that may have
%aborted for some reason  (This implies, of course, that the previous
%calculation employed the keyword \verb+chkpt+ with a value specified for
%the variable \verb+minutes+, which allowed the calculation to write out the
%gradient before failing).  The keyword \verb+restart+ allows the partially
%calculated forces from the previous calculation to be used as the starting
%point for the new calculation.  If the gradient was not saved previously,
%however, this keyword has no effect, and the gradients are automatically 
%recalculated from zero.

%  This keyword tells the program that this is a restart of an aborted
%  gradient calculation. The partially calculated forces are taken from
%  the database of the previous run. If they are not present, the
%  keyword is ignored and a complete calculation of the gradients is started.

%  This also works within a geometry optimization. Once a gradient
%calculation has completed, restart information is automatically
%deleted
%so that subsequent gradient 
%  calculations are not treated as restarts.
%RJH: I don't get the significance of this paragraph.  Elaborate on it?
%Is it saying that gradient calculations can be saved and restarted even
%if you are in the middle of a geometry optimization?  But once a
%gradient is completely calculated, the data that got you there is
%erased, even if you do have a chkpt directive statement?--fmr.

% \section{PRINT, NOPRINT}

The complementary keyword pair \verb+print+ and \verb+noprint+ allows
the user some additional control on the information that can be
included in the print output from the SCF calculation.  Currently,
only a few items can be explicitly invoked via print control.  These
are as follows:
 
%  Currently only some print control is available.

\begin{table}[htbp]
\begin{center}
\begin{tabular}{lcc}
  {\bf Name}       & {\bf Print Level} & {\bf Description} \\
 ``information''   &       low         & calculation info\\
 ``geometry''      &      high         & geometry information\\
 ``basis''         &      high         & basis set(s) used\\
 ``forces''        &       low         & details of force components\\
 ``timing''        &     default       & timing for each phase\\
\end{tabular}
\end{center}
\caption{Gradient Print Control Specifications}
\end{table}



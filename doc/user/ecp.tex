\label{sec:ecp}
\begin{verbatim}
  ecp [<string name default "ecp basis">] \
        [spherical||cartesian] \
        [segment||nosegment] \
        [print]

     <string tag> library [<string tag_in_lib>] \
                  <string standard set> [file <filename>]

         or

     <string tag> <string nelec> <integer number of electrons replace>
 
         or

     <string tag> <string shell_type>
     <real r-exponent> <real gaussian-exponent> <real list_of_coefficients>
        ...
     
  END
\end{verbatim}    

This directive describes an effective core potential (ECP) basis set
of contracted gaussian functions.  These are fit to gaussians by the
function:
\[
r^2V_l(r) = \sum_{k} A_{lk} r^{n_{lk}} e^{B_{lk}r^{2}}
\]
Where $A_{lk}$ is the contraction coefficient, $n_{lk}$ is the
exponent of the ``r'' term (r-exponent), and $B_{lk}$ is the gaussian
exponent.  The r-exponent is shifted by 2 as per most of the ECP
literature, e.g., an r-exponent of 0 implies $r^{-2}$.

By default basis sets are automatically segmented and cartesian even
if general contractions are input.  Generally contracted ECP basis
sets are not in wide use but the functionality is available.  ECP
basis functions are associated with centers in geometries through the
tags or names of centers which must match exactly (including case) and
are limited to sixteen (16) characters.  Each center with the same tag
will have the same ecp basis set.  By default the input module prints
each ecp basis set encountered; use the \verb+NOPRINT+ option to
disable printing.  There can be only one active ECP basis set even
though several may exist in the input deck.  The ECP modules load
``ecp basis'' with any ``ao basis'' present.  The ECP functionality
works for energy and gradients.

In the same fashion as for geometries or regular basis sets, ecp basis
sets are named, with the default name being \verb+"ecp basis"+.  It
should be clear from the above discussion on geometries and database
entries how indirection is supported.

Basis functions currently may not be drawn from a standard set in the
EMSL basis set library; they must be specified explicitly.  All
directives that are in common with the standard gaussian basis set
input have the same function and syntax.  

Che \verb+NELEC+ directive is the number of core electrons replaced by
the ecp basis specification for the atom represented by the tag.  

The shell type label of ``ul'' is used to describe the local part of
the ECP basis.  This is equivalent to the highest angular momentum
functions specified in the literature for most ecp basis sets.  The
standard shell type delineates the angular momentum projector onto the
local function.  The shell type label of ``s'' indicates the ``ul-s''
projector input, ``p'' indicates the ``ul-p'', etc.  See the example
below for clarification. 

{\bf Note:} currently the atomic SCF code does not handle the guess
generation for ECP centers.  Therefore, you may have trouble getting
your initial set of orbitals for either the SCF or DFT.  You may need
to use a smaller basis set and then project those orbitals to the
basis set that will be used, (c.f. \ref{sec:vectors}).  Convergence
and starting orbital guesses are being addressed.

The following example illustrate the input of an ECP for H$_2$CO.

\centerline{{\bf H$_2$CO }}

\begin{verbatim}
ecp print  
C nelec 2     # ecp replaces 2 electrons on C
C ul    # d
        1       80.0000000       -1.60000000
        1       30.0000000       -0.40000000
        2        0.5498205       -0.03990210
C s     # s - d 
        0        0.7374760        0.63810832
        0      135.2354832       11.00916230
        2        8.5605569       20.13797020
C p     # p - d
        2       10.6863587       -3.24684280
        2       23.4979897        0.78505765
O nelec 2     # ecp replaces 2 electrons on O
O ul    # d 
        1       80.0000000       -1.60000000
        1       30.0000000       -0.40000000
        2        1.0953760       -0.06623814
O s     # s - d
        0        0.9212952        0.39552179
        0       28.6481971        2.51654843
        2        9.3033500       17.04478500
O p     # p - s 
        2       52.3427019       27.97790770
        2       30.7220233      -16.49630500
end
\end{verbatim}


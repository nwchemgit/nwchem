\label{sec:driver}

The DRIVER module is one of two drivers (see Section \ref{sec:stepper}
for documentation on STEPPER) to perform a geometry optimization
function on the molecule defined by input using the \verb+GEOMETRY+
directive (see Section \ref{sec:geom}).  Geometry optimization is
either an energy minimization or a transition state optimization.
The algorithm programmed in DRIVER is a quasi-newton optimization
with line searches and approximate energy hessian updates.

DRIVER is selected by default out of the two available modules to
perform geometry optimization.  In order to force use of DRIVER (e.g.,
because a previous optimization used STEPPER) provide a DRIVER input
block (below) --- even an empty block will force use of DRIVER.

Optional input for this module is specified within the compound
directive,
\begin{verbatim}
  DRIVER 
    ...
  END
\end{verbatim}

Input specified for the DRIVER module may appear anywhere in the input
file preceding the \verb+TASK+ directive performing the optimization.
In the current version of NWChem, DRIVER uses geometries that are
defined via Cartesian coordinates or internal coordinates. The latter
may be user-defined with the Z-matrix input (Section
\ref{sec:Z-matrix}) or may be automatically generated using the
\verb+AUTOZ+ option (Section \ref{sec:geomkeys}).  The initial guess
nuclear Hessian is the identity matrix and there is an ASCII interface
file to input a Hessian from another code.  The automatic generation
of finite difference Hessians will be available soon. When internal
coordinates are selected an appropriate initial hessian matrix is
automatically created to match the internal coordinate definition.

Execution of the DRIVER module calculation is invoked with a
\verb+TASK+ directive (see Section \ref{sec:task}).

No input is required for DRIVER.  If no input is present the default
actions are to minimize the energy as a function of the geometry with a
maximum of 20 stepper iterations.

\section{{\tt NTPOPT} --- Maximum number of steps}

\begin{verbatim}
  NTPOPT  <integer nptopt  default 20>
\end{verbatim}

The value specified for the integer \verb+<ntpopt>+ defines the maximum 
number of geometry optimization steps.  

\section{{\tt CVGOPT} --- Convergence criterion}

\begin{verbatim}
  CVGOPT  <real cvgopt  default 0.0008>
\end{verbatim}
 
The value specified for the real \verb+<cvgopt>+ defines the convergence
threshold of the optimization algorithm. The convergence criterion is
the largest component of the energy gradient for the coordinates used
in the optimization ( cartesian or internal coordinates ).

\section{{\tt LINOPT} --- Linear search}

\begin{verbatim}
  LINOPT  <integer linopt  default 10>
\end{verbatim}

The value specified for the integer \verb+<linopt>+ defines the maximum 
number of energy points during any linear search. 

\section{{\tt INHESS} and {\tt MODUPD} --- Hessian update algorithm}

\begin{verbatim}
  INHESS  <integer inhess  default 0>
\end{verbatim}

\begin{verbatim}
  MODUPD  <integer modupt  default 1>
\end{verbatim}

The value specified for the integer \verb+<modupd>+ defines the hessian 
update algorithm, Fletcher-Powell update (\verb+<modupd> = 0+) or
Broyden-Fletcher-Goldfar-Shanno update (\verb+<modupd> =  1+)

\section{{\tt MODSAD} --- Optimization mode}

\begin{verbatim}
  MODSAD  <integer modsad  default 0>
\end{verbatim}

The value specified for the integer \verb+modsad+ defines the type   
of optimization to be performed, an energy minimization
(\verb+<modsad> = 0+)
or a transition state optimization (\verb+<modsad> = 1+).

\section{{\tt MODDIR} --- Normal mode selection}

\begin{verbatim}
  MODDIR  <integer moddir  default 1>
\end{verbatim}

The value specified for the integer \verb+moddir+ defines    
which normal mode of displacement to follow initially in the
transition state search.

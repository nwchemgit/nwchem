\label{sec:mcscf}
\begin{verbatim}
  mcscf
    ...
  end
\end{verbatim}

This compound directive provides input for the MCSCF module.
Currently, only Complete Active Space SCF (CASSCF) wavefunctions are
supported. The MCSCF module should ideally be invoked from a previously
converged RHF/ROHF SCF calculation using \verb+RESTART+
directive. The following input options have the same meaning
and syntax as with the SCF directive (\ref{SCF}); \verb+maxiter+,
\verb+thresh+, \verb+tol2e+ and \verb+vectors+.  These options default to
their corresponding SCF values if available, otherwise the SCF defaults
listed in section \ref{SCF} are used. The converged MCSCF molecular
orbitals and ``orbital energies'' are stored in ``canonical''
representation with the \verb+vectors output+ option.

The following three directives; \verb+active+, \verb+actelec+ and
\verb+multiplicity+ are mandatory and together with the charge specify
the CASSCF wavefunction. The inactive and secondary orbitals are
determined by reconciling the number of electrons and basis
functions. Currently, there is no facility to designate frozen core or
virtual orbitals. Although symmetry is used in various components, the
MCSCF solution is not symmetry blocked and symmetry-forbidden
rotations are quietly screened.

The CI submodule is determinant-based consequently L\"owdin spin
projection is used for low-spin states. Moreover, the CI vector is
not symmetry blocked but the correct state irreducible representation
is maintained by symmetry projection.

\subsection{Inactive}
\begin{verbatim}
  inactive <integer inactive>
\end{verbatim}
Specifies the lowest \verb+inactive+ orbitals as inactive.

\subsection{Active}
\begin{verbatim}
  active <integer active>
\end{verbatim}
Designate the next \verb+active+ orbitals after the \verb+inactive+
orbitals as active. 

\subsection{Active Electrons}
\begin{verbatim}
  actelec <integer actelec>
\end{verbatim}
Sets the number of electrons, $N_{a}$, in the active space.

\subsection{Multiplicity}
\begin{verbatim}
  multiplicity <integer multiplicity>
\end{verbatim}
Sets the spin multiplicity of the wavefunction, $(2S + 1)$. The number
of active alpha and beta electrons is determined by using $M_{s} =
(n_{\alpha} - n_{\beta})/2 = S$ and $N_{a} = n_{\alpha} + n_{\beta}$.

\subsection{Symmetry}
\begin{verbatim}
  symmetry <integer state_irrep default 1>
\end{verbatim}
Sets the desired state symmetry.

\subsection{Level Shift}
\begin{verbatim}
  level <real levelshift default 0.1>
\end{verbatim}
Specifies the initial level shift applied the orbital Hessian. Note
this also defaults to the SCF level shift if present. The level shift
remains constant at each macroiteration until the gradient norm is
less than $10^{-2}$~au whereupon it is rapidly reduced to zero.

\subsection{Profile}
\begin{verbatim}
  profile <string options>
\end{verbatim}
The directive by itself enables general profiling of MCSCF components
including orbital solution, CI solution and transformation. Additional
options; \verb+fock+ and \verb+ci+ enable more detailed profiling of
these components

\subsection{Hessian Type}
\begin{verbatim}
  hessian <string type default ``exact'' >
\end{verbatim}
This directive selects the type of orbital Hessian used in the
Newton-Raphson orbital solution. Possible choices are \verb+onel+ and
\verb+exact+. The \verb+onel+ option uses only Fock operators to
approximate the Hessian with first-order convergence. The \verb+exact+
option selects the exact Hessian which involves a Fock matrix
construction each microiteration. 

\begin{table}
\caption{MCSCF variables}
\label{MCSCF_variables}
\vspace{.2in}
\begin{tabular}{lrrl}
\hline\hline
Variable                        & Type     & Default          & Synopsis \\
\hline
\verb+mcscf:e2approx+           & logical  &  TRUE            & Use E2(T) extrapolation \\
\verb+mcscf:microci+            & logical  &  TRUE            & CI relaxed within line search \\
\verb+mcscf:conjugacy+          & logical  &  TRUE            & Conjugacy used \\
\verb+mcscf:cgreset+            & integer  &  -1              & Interval between conjugacy resets \\
\verb+mcscf:canonical+          & logical  &  TRUE            & Generate and save canonical orbitals \\
\verb+mcscf:movecs lock+        & logical  &  FALSE           & Lock MO ordering to previously saved \\
\verb+mcscf:citol+              & real     &  1.d-8           & CI convergence tolerance \\
\verb+mcscf:line_search_tol+    & real     &  0.1             & Line search tolerance  \\
\verb+mcscf:ciiterlo+           & integer  &  1               & Macro CI relax low end \\
\verb+mcscf:ciiterhi+           & integer  &  \verb+maxiter+  & Macro CI relax high end \\
\hline\hline
\end{tabular}
\end{table}

\subsection{Variables and Print Control}
Greater control of the MCSCF module is provided by several variables
listed in the table~(\ref{MCSCF_variables}). In most instances, the
defaults should suffice. For some exceptional cases, these variables can
be modified using the \verb+SET+ syntax (see \ref{sec:set}).

The variables, \verb+ciiterlo+ and \verb+ciiterhi+, specify the
macroiteration range where the CI wavefunction is optimized. By
default, CI wavefunction optimization is performed at every
macroiteration. Furthermore, unless \verb+ciiterlo+ is set to zero,
the very first CI optimization is {\em always} performed in order to
generate an initial density. While setting \verb+ciiterlo+ to zero
forces an high-spin ROHF calculation. A possible convergence strategy
is to set the \verb+ciiterlo+ to about 5 to ensure a tight
optimization of the inactive orbitals.

Output items can selectively enabled or disabled using the
\verb+print+ control mechanism~(\ref{sec:printcontrol}) with the
available print options listed in table(\ref{MCSCF_print_options}).



\begin{table}
\caption{MCSCF Print Options}
\label{MCSCF_print_options}
\vspace{.2in}
\begin{tabular}{lrl}
\hline\hline
Option                          & Class    &  Synopsis \\
\hline
\verb+ci energy+                & default  &  CI energy eigenvalue \\
\verb+fock energy+              & default  &  Energy derived from Fock matrices \\
\verb+gradient norm+            & default  &  Gradient norm \\
\verb+movecs+                   & default  &  Converged occupied MO vectors \\
\verb+trace energy+             & high     &  Trace Energy \\
\verb+converge info+            & high     &  Convergence data and monitoring \\
\verb+precondition+             & high     &  Orbital preconditioner iterations \\
\verb+microci+                  & high     &  CI iterations in line search \\
\verb+canonical+                & high     &  Canonicalization infomation \\
\verb+new movecs+               & debug    &  MO vectors at each macroiteration \\
\verb+ci guess+                 & debug    &  Initial guess CI vector \\
\verb+density matrix+           & debug    &  One- and Two-particle density matrices \\
\hline\hline
\end{tabular}
\end{table}

%
% $Id$
%
\label{sec:mcscf}

The NWChem multiconfiguration SCF (MCSCF) module can currently perform
complete active space SCF (CASSCF) calculations with at most 20 active
orbitals and about 500 basis functions.  It is planned to extend it to 
handle 1000+ basis functions.

\begin{verbatim}
  MCSCF
    STATE <string state>
    ACTIVE <integer nactive>
    ACTELEC <integer nactelec>
    MULTIPLICITY <integer multiplicity>
    [SYMMETRY <integer symmetry default 1>]
    [VECTORS [[input] <string input_file default $file_prefix$.movecs>] 
           [swap <integer vec1 vec2> ...] \
           [output <string output_file default input_file>] \
           [lock]
    [HESSIAN (exact||onel)]
    [MAXITER <integer maxiter default 20>]
    [THRESH  <real thresh default 1.0e-4>]
    [TOL2E <real tol2e default 1.0e-9>]
    [LEVEL <real shift default 0.1d0>]
  END
\end{verbatim}
Note that the \verb+ACTIVE+, \verb+ACTELEC+, and \verb+MULTIPLICITY+
directives are {\em required}.  The symmetry and multiplicity may
alternatively be entered using the \verb+STATE+ directive.

\section{{\tt ACTIVE} --- Number of active orbitals}

The number of orbitals in the CASSCF active space must be specified
using the {\tt ACTIVE} directive.

E.g.,
\begin{verbatim}
  active 10
\end{verbatim}

The input molecular orbitals (see the vectors directive, Sections
\ref{sec:mcscfvectors} and \ref{sec:vectors}) must be arranged in
order
\begin{enumerate}
\item doubly occupied orbitals,
\item active orbitals, and
\item unoccupied orbitals.
\end{enumerate}

\section{{\tt ACTELEC} --- Number of active electrons}

The number of electrons in the CASSCF active space must be specified
using the the {\tt ACTELEC} directive.  An error is reported if the
number of active electrons and the multiplicity are inconsistent.

The number of closed shells is determined by subtracting the number
of active electrons from the total number of electrons (which in turn
is derived from the sum of the nuclear charges minus the total system
charge).

\section{{\tt MULTIPLICITY}}

The spin multiplicity must be specified and is enforced by projection
of the determinant wavefunction.

E.g., to obtain a triplet state
\begin{verbatim}
  multiplicity 3
\end{verbatim}

\section{{\tt SYMMETRY} --- Spatial symmetry of the wavefunction}

This species the irreducible representation of the wavefunction as an
integer in the range 1---8 using the same numbering of representations
as output by the SCF program.  Note that only Abelian point groups are
supported.

E.g., to specify a $B_1$ state when using the $C_{2v}$ group
\begin{verbatim}
  symmetry 3
\end{verbatim}

\section{{\tt STATE} --- Symmetry and multiplicity}

The electronic state (spatial symmetry and multiplicity) may
alternatively be specified using the conventional notation for an
electronic state, such as $^3B_2$ for a triplet state of $B_2$
symmetry.  This would be accomplished with the input
\begin{verbatim}
  state 3b2
\end{verbatim}
which is equivalent to 
\begin{verbatim}
  symmetry 4
  multiplicity 3
\end{verbatim}

\section{{\tt VECTORS} --- Input/output of MO vectors}
\label{sec:mcscfvectors}

Calculations are best started from RHF/ROHF molecular orbitals (see
Section \ref{sec:scf}), and by default vectors are taken from the
previous MCSCF or SCF calculation.  To specify another input file use
the \verb+VECTORS+ directive.  Vectors are by default output to the
input file, and may be redirected using the \verb+output+ keyword.
The \verb+swap+ keyword of the \verb+VECTORS+ directive may be
used to reorder orbitals to obtain the correct active space.
See Section \ref{sec:vectors} for an example.

The \verb+LOCK+ keyword allows the user to specify that the ordering
of orbitals will be locked to that of the initial vectors, insofar as
possible. The default is to order by ascending orbital energies within
each orbital space. One application where locking might be desirable
is a calculation where it is necessary to preserve the ordering of a
previous geometry, despite flipping of the orbital energies.  For such
a case, the \verb+LOCK+ directive can be used to prevent the SCF
calculation from changing the ordering, even if the orbital energies
change.

Output orbitals of a converged MCSCF calculation are canonicalized as
follows:
\begin{itemize}
\item Doubly occupied and unoccupied orbitals diagonalize the
  corresponding blocks of an effective Fock operator.  Note that in
  the case of degenerate orbital energies this does not fully
  determine the orbtials.
\item Active-space orbitals are chosen as natural orbitals by
  diagonalization of the active space 1-particle density matrix.
  Note that in the case of degenerate occupations that this
  does not fully determine the orbitals.
\end{itemize}

\section{{\tt HESSIAN} --- Select preconditioner}
\label{sec:mcscfhessian}

The MCSCF will use a one-electron approximation to the orbital-orbital
Hessian until some degree of convergence is obtained, whereupon it
will attempt to use the exact orbital-orbital Hessian which makes the
micro iterations more expensive but potentially reduces the total
number of macro iterations.  Either choice may be forced throughout
the calculation by specifying the appropriate keyword on the
\verb+HESSIAN+ directive.

E.g., to specify the one-electron approximation throughout
\begin{verbatim}
  hessian onel
\end{verbatim}

\section{{\tt LEVEL} --- Level shift for convergence}

The Hessian (Section \ref{sec:mcscfhessian}) used in the MCSCF
optimization is by default level shifted by 0.1 until the orbital
gradient norm falls below 0.01, at which point the level shift is
reduced to zero.  The initial value of $0.1$ may be changed using
the \verb+LEVEL+ directive.  Increasing the level shift may make
convergence more stable in some instances.

E.g., to set the initial level shift to 0.5
\begin{verbatim}
  level 0.5
\end{verbatim}

\section{{\tt PRINT} and {\tt NOPRINT}}

Specific output items can be selectively enabled or disabled using the
\verb+print+ control mechanism~(\ref{sec:printcontrol}) with the
available print options listed in table(\ref{MCSCF_print_options}).

\begin{table}[htb]

\label{MCSCF_print_options}

\center

\vspace{.2in}
\begin{tabular}{lrl}
\hline\hline
Option                          & Class    &  Synopsis \\
\hline
\verb+ci energy+                & default  &  CI energy eigenvalue \\
\verb+fock energy+              & default  &  Energy derived from Fock matrices \\
\verb+gradient norm+            & default  &  Gradient norm \\
\verb+movecs+                   & default  &  Converged occupied MO vectors \\
\verb+trace energy+             & high     &  Trace Energy \\
\verb+converge info+            & high     &  Convergence data and monitoring \\
\verb+precondition+             & high     &  Orbital preconditioner iterations \\
\verb+microci+                  & high     &  CI iterations in line search \\
\verb+canonical+                & high     &  Canonicalization information \\
\verb+new movecs+               & debug    &  MO vectors at each macro-iteration \\
\verb+ci guess+                 & debug    &  Initial guess CI vector \\
\verb+density matrix+           & debug    &  One- and Two-particle density matrices \\
\hline\hline
\end{tabular}

\caption{MCSCF Print Options}

\end{table}



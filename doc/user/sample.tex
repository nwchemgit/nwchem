\label{sec:sample}
\subsection{Water SCF calculation and geometry optimization in a 6-31g basis}

The input file in section \ref{sec:getstart} performs a geometry optimization
in a single task.  Here we perform a single point SCF energy calculation and then
restart it to perform the optimization.

\subsection{Job 1.  Single point SCF energy}

\begin{verbatim}
  start h2o
  title; Water in 6-31g basis set

  geometry
    O      0.00000000    0.00000000    0.00000000
    H      0.00000000    1.43042809   -1.10715266
    H      0.00000000   -1.43042809   -1.10715266
  end
  basis print
    H library 6-31g
    O library 6-31g
  end
  task scf
\end{verbatim}

The final energy should be -75.9839975707.

\subsection{Job 2. Restarting and perform a geometry optimization}

\begin{verbatim}
  restart h2o
  title; Water geometry optimization

  task scf optimize
\end{verbatim}

There is no need to specify anything that has not changed from the
previous input deck, though it will do no harm to repeat it.  The
final energy and geometry should be $-75.9853591759$, O
$(0,0,0.1563305320)$, and H $(0, \pm1.48372809, -0.853122128)$.

\subsection{Compute the polarizability of Ne using finite field}

\subsection{Job 1. Compute the atomic energy}

\begin{verbatim}
  start ne
  title; Neon
  geometry; ne 0 0 0; end
  basis spherical 
    ne library aug-cc-pvdz
  end
  scf; thresh 1e-10; end
  task scf
\end{verbatim}

The final energy should be -128.49634973.

\subsection{Job 2. Compute the energy with applied field}

An external field may be simulated with point charges.  The charges
here apply a field of magnitude 0.01\ atomic units to the atom at the
origin.  Since the basis functions have not been reordered by the
additional centers we can also restart from the previous vectors,
which is the default for a restart job.

\begin{verbatim}
  restart ne
  title; Neon in electric field
  geometry
    bq1 0 0 100 charge 50
    ne  0 0 0
    bq2 0 0 -100 charge -50
  end
  task scf
\end{verbatim}

The final energy should be -128.49644133, which together with the
previous field-free result yields an estimate for the polarizability
of 1.83 atomic units.  Note that by default NWChem does not include
the interaction between the two point charges in the total energy
(section \ref{sec:geom}).


%\chapter{Dynamic Nucleation Theory Monte Carlo}

% $Id: dntmc.tex,v 1.1 2008-03-04 17:48:26 windus Exp $
%
\label{sec:dntmc} 1.)  Schenter, G. K.; Kathmann, S. M.; Garrett, B.
C. {\it J. Chem. Phys.} ({\bf 1999}), {\it 110}, 7951. 2.)Crosby, L.
D.; Kathmann, S. M.; Windus, T. L. ({\bf 2008}) in preparation.

 The Dynamical Nucleation Theory/Monte Carlo
(DNTMC) module utilizes Dynamical Nucleation Theory to compute monomer
evaporation rate constants at a given temperature.  The reactant is
a molecular cluster of $i$ rigid monomers while the product is a
molecular cluster with $i-1$ monomers plus a free monomer.  A
Metropolis Monte Carlo methodology is utilized to sample the
configurational space of these $i$ rigid monomers.  At present only
homogenous clusters are supported by this module.  The input does
allow for the definition of heterogeneous clusters, however, the
underlying code does not correctly extend to these systems.

\section{SubGroups}

The DNTMC module supports the use of subgroups in the MC
simulations.  The number of subgroups is defined in the input
through a set directive:
\begin{verbatim}
     set subgroup_number <integer number>
\end{verbatim}
,where the number of subgroups requested is the argument.  The
number of processors that each subgroup has access to is determined
by Total/subgroup\_number.  A separate MC simulation is performed
within each subgroup.  To use this functionality, NWChem must be
compiled with the USE\_SUBGROUPS environmental variable set to yes.

Each MC simulation starts at a different starting configuration,
which is equally spaced along the reaction coordinate.  The
statistical distributions which these MC simulations produce are
averaged to form the final statistical distribution.  Output from
these subgroups consists of various files whose names are of the
form (*.\#num). These files include restart files and other data
files.  The RTDB is used as input for these subgroups and must be
globally accessible (set through the Permanent\_Dir directive) to
all processes.

\newpage
\section{Input Syntax}

The input block has the following form:

\begin{verbatim}
DNTMC
     [nspecies <integer number>]
     [species  <list of strings name[nspecies]]
     [nmol     <list of integers number[nspecies]>]

     [temp     <real temperature>]
     [rmin     <real rmin>]
     [rmax     <real rmax>]
     [mcsteps  <integer number>]
     [tdisp    <real disp>]
     [rdisp    <real rot>]
     [rsim || rconfig]
     [mprnt    <integer number>]
     [convergence <real limit>]
     [norestart]
     [dntmc_dir <string directory>]

     [print &&|| noprint]

     [procrestart <integer number>]
END
\end{verbatim}

\section{Definition of Monomers}

Geometry informaion is required for each unique monomer (species).
See the geometry input section \ref{sec:geom} for more information.
A unique label must be given for each monomer geometry.

\begin{verbatim}
  GEOMETRY [<string name species_1>] ...
      ...
  END

  GEOMETRY [<string name species_2>] ...
      ...
  END

  ...
\end{verbatim}

The molecular cluster is defined by the number of unique monomers
(nspecies).  The geometry labels for each unique monomer is given in
a space delimited list (species).  Also required are the number of
each unique monomer in the molecular cluster given as a space
delimited list (nmol).  These keywords are required and thus have no
default values.

\begin{verbatim}
     [nspecies <integer number>]
     [species  <list of strings name[nspecies]]
     [nmol     <list of integers number[nspecies]>]
\end{verbatim}

\section{DNTMC runtime options}

Several options control the behavior of the DNTMC method.  Some
required options such as simulation temperature (temp), cluster
radius (rmin and rmax), and maximum number of Monte Carlo steps
(mcsteps) are used to control the MC simulation.

\begin{verbatim}
     [temp     <real temperature>]
\end{verbatim}
This required option gives the simulation temperature in which the
MC simulation is run.  Temperature is given in kelvin.

\begin{verbatim}
     [rmin     <real rmin>]
     [rmax     <real rmax>]
\end{verbatim}
These required options define the minimum and maximum extent of the
projected reaction coordinate (The radius of a sphere centered at
the center of mass).  Rmin should be large enough to contain the
entire molecular cluster of monomers and Rmax should be large enough
to include any relevant configurational space (such as the position
of the reaction bottleneck).  These values are given in Angstroms.

\begin{verbatim}
     [mcsteps  <integer number>]
     [tdisp    <real disp default 0.04>]
     [rdisp    <real rot default 0.06>]
     [convergence <real limit default 0.00>]
\end{verbatim}
These options define some characteristics of the MC simulations. The
maximum number of MC steps (mcsteps) to take in the course of the
calculation run is a required option.  Once the MC simulation has
performed this number of steps the calculation will end.  The
maximum translational step size (tdisp) and rotational step size
(rdisp) are optional inputs with defaults set at 0.04 Angstroms and
0.06 radians, respectively.  The convergence keyword allows the
convergence threshold to be set.  The default is 0.00 which
effectively turns off this checking.  Once the measure of
convergence goes below this threshold the calculation will end.

\begin{verbatim}
     [rsim || rconfig]
\end{verbatim}
These optional keywords allow the selection of two different MC
sampling methods.  rsim selects a Metropolis MC methodology which
samples configurations according to a Canonical ensemble.  The
rconfig keyword selects a MC methodology which samples
configurations according to a derivative of the Canonical ensemble
with respect to the projected reaction coordinate.  These keywords
are optional with the default method being rconfig.

\begin{verbatim}
     [mprnt    <integer number default 10>]
     [dntmc_dir <string directory default ./>]
     [norestart]
\end{verbatim}
These three options define some of the output and data analysis
behavior. mprnt is an option which controls how often data analysis
occurs during the simulation.  Currently, every mprnt*100 MC steps
data analysis is performed and results are output to files and/or to
the log file.  Restart files are also written every mprnt number of
MC steps during the simulation.  The default value is 10. The
keyword dntmc\_dir allows the definition of an alternate directory
to place DNTMC specific ouputfiles.  These files can be very large
so be sure enough space is available.  This directory should be
accessible by every process (although not necessarily globally
accessible).  The defualt is to place these files in the directory
which NWChem is run (./).  The keyword norestart turns off the
production of restart files. By default restart files are produced
every mprnt number of MC steps.

\section{Print Control}
The DNTMC module supports the use of PRINT and NOPRINT Keywords. The
specific labels which DNTMC recognizes are included below.

\begin{table}[htbp!]
\begin{center}
\begin{tabular}{lcc}
  {\bf Name}          & {\bf Print Level} & {\bf Description} \\
``debug'' &      debug & \begin{minipage}{0.6\textwidth}Some debug
information written in Output
file.\end{minipage} \\
\\
``information'' & none & \begin{minipage}{0.6\textwidth}Some
information such as energies and
geometries.\end{minipage}\\
\\
``mcdata'' & low & \begin{minipage}{0.6\textwidth} Production of a
set of files (Prefix.MCdata.\#num). These files are a concatenated
list of structures, Energies, and Dipole Moments for each accepted
configuration sampled in the MC run.\end{minipage}\\
\\
``alldata'' & low & \begin{minipage}{0.6\textwidth}Production of a
set of files (Prefix.Alldata.\#num).  These files include the same
information as MCdata files.  However, they include ALL
configurations (accepted or
rejected).\end{minipage}\\
\\
``mcout'' & debug -- low  &
\begin{minipage}{0.6\textwidth}Production of a set of files
(Prefix.MCout.\#num).  These files contain a set of informative and
debug information.  Also included is the set of information which
mirrors the Alldata files.\end{minipage}\\
\\
``fdist'' & low & \begin{minipage}{0.6\textwidth}Production of a
file (Prefix.fdist) which contains
a concatenated list of distributions every mprnt*100 MC steps.\end{minipage}\\
\\
\end{tabular}
\end{center}
\end{table}

\section{DNTMC Restart}

\begin{verbatim}
     [procrestart <integer number>]
\end{verbatim}
Flag to indicate restart postprocessing.  In order to restart a
DNTMC run postprocessing is required to put required information
into the runtime database (RTDB).  During a run restart information
is written to files (Prefix.restart.\#num) every mprnt MC steps.
This information must be read and deposited into the RTDB before a
restart run can be done.  The number taken as an argument is the
number of files to read and must also equal the number of subgroups
the calculation utilizes.  The start directive must also be set to
restart for this to work properly.  All input is read as usual.
However, values from the restart files take precedence over input
values. Some keywords such as mcsteps are not defined in the restart
files.  Task directives are ignored. You must have a RTDB present in
your permanent directory.

Once postprocessing is done a standard restart can be done from the
RTDB by removing the procrestart keyword and including the restart
directive.

\section{Task Directives}
The DNTMC module can be used with any level of theory which can
produce energies.  Gradients and Hessians are not required within
this methodology.  If dipole moments are available, they are also
utilized.  The task directive for the DNTMC module is shown below:
\begin{verbatim}
     task <string theory> dntmc
\end{verbatim}

\section{Example}
This example is for a molecular cluster of 10 monomers.  A 50/50
mixture of water and ammonia.  The energies are done at the
SCF/6-31++G** level of theory.

\begin{verbatim}
    start
    # start or restart directive if a restart run
    MEMORY 1000 mb

    PERMANENT_DIR /home/bill
    # Globally accessible directory which the
    # rtdb (*.db) file will/does reside.

    basis "ao basis" spherical noprint
        * library 6-31++G**
    end
    # basis set directive for scf energies

    scf
        singlet
        rhf
        tol2e 1.0e-12
        vectors input atomic
        thresh 1.0e-06
        maxiter 200
        print none
    end
    # scf directive for scf energies

    geometry geom1 units angstroms noautosym nocenter noprint
    O  0.393676503613369      -1.743794626956820      -0.762291912129271
    H -0.427227157125777      -1.279138812526320      -0.924898279781319
    H  1.075463952717060      -1.095883929075060      -0.940073459864222
    symmetry c1
    end
    # geometry of a monomer with title "geom1"

    geometry geom2 units angstroms noautosym nocenter noprint
    N     6.36299e-08     0.00000     -0.670378
    H     0.916275     0.00000     -0.159874
    H     -0.458137     0.793517     -0.159874
    H     -0.458137     -0.793517     -0.159874
    symmetry c1
    end
    # geometry of another monomer with title "geom2"
    # other monomers may be included with different titles

    set subgroups_number 8
    # set directive which gives the number of subgroups
    # each group runs a separate MC simulation

    dntmc
    # DNTMC input block
        nspecies 2
        # The number of unique species (number of titled geometries
        # above)
        species geom1 geom2
        # An array of geometry titles (one for each
        # nspecies/geometry)
        nmol    5  5
        # An array stating the number of each
        # monomer/nspecies/geometry in simulation.
        temp  243.0
        mcsteps 1000000
        rmin 3.25
        rmax 12.25
        mprnt 10
        tdisp 0.04
        rdisp 0.06
        print none fdist mcdata
        # this print line first sets the print-level to none
        # then it states that the *.fdist and *.mcdata.(#num)
        # files are to be written
        rconfig
        dntmc_dir /home/bill/largefile
        # An accessible directory which to place the *.fdist,
        # *.mcdata.(#num), and *.restart.(#num) files.
        convergence 1.0D+00
    end

    task scf dntmc
    # task directive stating that energies are to be done at the scf
    #level of theory.
\end{verbatim}


\label{sec:basis}
\begin{verbatim}
  BASIS [<string name default "ao basis">] \
        [spherical||cartesian] \
        [segment||nosegment] \
        [print]

     <string tag> library [<string tag_in_lib>] \
                  <string standard set> [file <filename>]

         or

     <string tag> <string shell_type>
        <real exponent> <real list_of_coefficients>
        ...
     
  END
\end{verbatim}    

This directive describes a basis set of generally contracted Cartesian
Gaussian functions.  By default basis sets are automatically segmented
and cartesian\footnote{This will change when the derivative integral
  code is extended to handle general contractions and sphericals. Only
  the energy codes can presently handle these.} even if general
contractions are input.  Basis functions are associated with centers
in geometries through the tags or names of centers which must match
exactly (including case) and are limited to sixteen (16) characters.
Each center with the same tag will have the same basis set.  By
default the input module prints each basis set encountered; use the
\verb+NOPRINT+ option to disable printing.

In the same fashion as for geometries, basis sets are named, with the
default name being \verb+"ao basis"+.  It should be clear from the
above discussion on geometries and database entries how indirection is
supported.

Basis functions may be either drawn from a standard set in the EMSL
basis set library or specified explicitly.  See Appendix
\ref{knownbasis} for the current list of standard basis sets. An
installation default is provided for the path to the basis set
library, which may be overridden in the input.  The case of names of
standard basis sets is ignored.  By default the standard basis
appropriate to the atom on which it is centered is used, however this
may be overriden, for instance to place ghost functions on dummy
centers in counterpoise calculations.  When explicitly specifying
generally contracted basis sets, simply specify on each line the
exponent followed by the contraction coefficients for each contraction
of that exponent.  The following examples will make the use of this
directive clear.

This directive uses Thom Dunning's cc-pvdz basis for oxygen and
hydrogen and augments it with a diffuse {\em s} on oxygen.
The default name of \verb+"ao basis"+ is used and it will be printed.
\begin{verbatim}
  basis
    oxygen library cc-pvdz
    hydrogen library cc-pvdz
    oxygen s
      0.01 1.0
  end
\end{verbatim}

The following directive specifies exactly the same basis set as the
previous directive except that all basis functions are explicitly
described (all input is free format --- the formatting here is just
for readability).
\begin{verbatim}
  basis
    oxygen s
      11720.0000    0.000710  -0.000160
       1759.0000    0.005470  -0.001263
        400.8000    0.027837  -0.006267
        113.7000    0.104800  -0.025716
         37.0300    0.283062  -0.070924
         13.2700    0.448719  -0.165411
          5.0250    0.270952  -0.116955
          1.0130    0.015458   0.557368
          0.3023   -0.002585   0.572759
    oxygen s                
          0.3023    1.000000
    oxygen p                
         17.7000    0.043018
          3.8540    0.228913
          1.0460    0.508728
          0.2753    0.460531
    oxygen p                
          0.2753    1.000000
    oxygen d
          1.1850    1.000000
    hydrogen s
         13.0100    0.019685
          1.9620    0.137977
          0.4446    0.478148
          0.1220    0.501240
    hydrogen s  
          0.1220    1.000000
    hydrogen p  
          0.7270    1.000000
    oxygen s
          0.01      1.0
  end
\end{verbatim}
Note that the correlation-consistent basis sets were designed using
spherical harmonics and to use these the \verb+SPHERICAL+ keyword must
be present on the \verb+BASIS+ directive.
    
This example uses a 3-21g basis set for centers \verb+o+ and \verb+si+
and explicitly specifies the path to the basis set library (perhaps
because you cannot find the copy installed with NWChem).
\begin{verbatim}
  basis
    o  library 3-21g file /usr/d3g681/nwchem/library
    si library 3-21g file /usr/d3g681/nwchem/library
  end
\end{verbatim}

In order to perform counterpoise corrections in standard basis sets it
is necessary to specify the atom type.  The following input specifies a
cc-pvdz basis for a calculation on the water monomer in the dimer basis.
\begin{verbatim}
  basis
    o   library cc-pvdz
    h   library cc-pvdz
    bqo library o cc-pvdz
    bqh library h cc-pvdz
  end
\end{verbatim}

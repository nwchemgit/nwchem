\label{sec:etrans}

The NWChem ET module uses the method of Corresponding Orbital Transformation to calculate the
electron transfer matrix element (or coupling) between ET reactants and product states,
$V_{rp}$ (sometimes called $H_{AB}$ or $V_{AB}$ in the literature). According to Marcus' two-state model $V_{rp}$ appears in the
expression for the electron transfer rate:

\begin{equation} 
{k_{ET}}=
\frac{2\pi}{\hbar}
V_{rp}^{2}
\frac{1}{\sqrt{4\pi \lambda k_{B}T}}
\exp \left( \frac{- \Delta G^{*}}{k_{B} T} \right)
\end{equation}


Suggested references are listed below.  The first reference gives a good basic description 
of the two-state model, and the appendix of the second reference details the method used
in the ET module.

\begin{enumerate}
\item J.R. Bolton, N. Mataga, and G. McLendon in ``Electron Transfer in Inorganic, Organic and Biological Systems"
(American Chemical Society, Washington, D.C., 1991)
\item A. Farazdel, M. Dupuis, E. Clementi, and A. Aviram, 
J.~Am.~Chem.~Soc., 112, 4206 (1990).
\end{enumerate}

\section{{\tt VECTORS} --- input of MO vectors for ET reactant and product states}
\label{sec:vectors}


\begin{verbatim}
  VECTORS [reactants] <string reactants_vectors_filename>
  VECTORS [products ] <string products_vectors_filename>
\end{verbatim}

The \verb+VECTORS+ directive allows the user to specify the source 
of the molecular orbital vectors for the ET reactant and product states. 
This is required input, as no default filename will be set by the program.
In fact, this is the only required input in the ET module, although there are
other optional keywords.



\section{{\tt FOCK/NOFOCK} --- method for calculating the 2e contribution to the coupling}
\label{sec:fock}

 \begin{verbatim}
   <string (FOCK||NOFOCK) default FOCK>
 \end{verbatim}

This directive enables/disables the use of the SCF Fock matrix 
routine in the calculation of the two-electron portion of the Hamiltonian.
Since the Fock matrix routine has been optimized for speed, accuracy and parallel performance,
it is the most efficient choice.

The user can calculate the two-electron contribution to the ET Hamiltonian
with another algorithm which may be more accurate for systems with a small
number of basis functions, although it is slower.

\section{{\tt TOL2E} --- integral screening threshold}
\label{sec:tol2e}

\begin{verbatim}
  TOL2E <real tol2e default max(10e-12,min(10e-7 , S(rp)*10e-7 )> 
\end{verbatim}

The variable \verb+tol2e+ is used in determining the integral
screening threshold for the evaluation of the two-electron contribution to the Hamiltonian
between the electron transfer reactant and product states.
As a default, \verb+tol2e+ is set depending on the magnitude
of the overlap between the ET reactant and product states ($S_{rp}$), within the range 1.0d-12 $\rightarrow$ 1.0d-7.

The input to specify the threshold explicitly within the \verb+ET+
directive is, for example:

\begin{verbatim}
  tol2e 1e-9
\end{verbatim}

\section{{\tt INPUT/OUTPUT EXAMPLE}}

The following example is for a simple electron transfer system, $He_{}$ $\rightarrow$ $He^{ +}$.
It is crucial that the transferring electron be localized in the ET reactant and product
wavefunctions resulting from their respective open-shell SCF or DFT calculations. This can be accomplished
using either a fragment guess (shown in the example, see \ref{sec:fragguess}), or a scaled density 
guess in which the charges on the atoms
are set in the atomic density (see \ref{sec:atomscf}).

Example input :
\begin{verbatim}
start he_et
title  "he...he(+) ET using fragment guess"
echo

geometry he
 He 0.0 0.0 0.0
 symmetry C1
end

geometry he2
 He 0.0 0.0  2.0
 He 0.0 0.0 -2.0
 symmetry C1
end

basis
 He   library 3-21G
end

set geometry he
charge 1
scf
 doublet; uhf; vectors input atomic output hep.mo
end
task scf energy

set geometry he
charge 0
scf
 singlet; uhf; vectors input atomic output he.mo
end
task scf energy

set geometry he2
charge 1
scf
 doublet; uhf; vectors input fragment hep.mo he.mo output hea.mo
end
task scf energy 

set geometry he2
charge 1
scf
 doublet; uhf; vectors input fragment he.mo hep.mo output heb.mo
end
task scf energy

et
 vectors reactants hea.mo 
 vectors products heb.mo
end
task scf et

\end{verbatim}
It is important to verify the localization of the electron in the calculation 
of the vectors in \verb+hea.mo+ and \verb+heb.mo+. To do this, carefully examine the Mulliken population
analysis.  For instance, for the ET reactant state in the He example, the Mulliken population
analysis looks like this:

\begin{verbatim}

   Mulliken analysis of the total density
   -------------------------------------

    Atom       Charge   Shell Charges
 -----------   ------   -------------------------------------------------------
    1 He   2     1.00   0.56  0.44
    2 He   2     2.00   0.78  1.22

  Mulliken analysis of the alpha density
  --------------------------------------

    Atom       Charge   Shell Charges
 -----------   ------   -------------------------------------------------------
    1 He   2     1.00   0.56  0.44
    2 He   2     1.00   0.39  0.61

   Mulliken analysis of the beta density
   -------------------------------------

    Atom       Charge   Shell Charges
 -----------   ------   -------------------------------------------------------
    1 He   2     0.00   0.00  0.00
    2 He   2     1.00   0.39  0.61

   Mulliken analysis of the spin density
   -------------------------------------

    Atom       Charge   Shell Charges
 -----------   ------   -------------------------------------------------------
    1 He   2     1.00   0.56  0.44
    2 He   2     0.00   0.00  0.00
\end{verbatim}
Clearly the electron is localized on the first He atom, and it has beta spin. If the fragment guess
or scaled atomic guess were not used, the spin density would be 0.5 on both He atoms, the overlap between
the ET reactant and product states would be \verb+100 %+ and an infinite
$V_{rp}$ would result.

Here is what the ET output looks like for this example:
\begin{verbatim}
                           Electron Transfer Calculation
                           -----------------------------

 MO vectors for reactants: hea.mo
 MO vectors for products : heb.mo

 Electronic energy of reactants     H(rr)      -5.3402392824
 Electronic energy of products      H(pp)      -5.3402392824

 Reactants/Products overlap         S(rp)      -0.0006033839

 Reactants/Products interaction energy:
 -------------------------------------
 One-electron contribution         H1(rp)       0.0040314092
 Two-electron contribution         H2(rp)      -0.0007837138
 Total interaction energy           H(rp)       0.0032476955

 Electron Transfer Coupling Energy, |Vrp|       0.0000254810
                                                       5.592 cm-1
                                                    0.000693 eV
                                                       0.016 kcal/mol
\end{verbatim}

The overlap between the ET reactant and product states ($S_{rp}$) is very small,
so the magnitude of the coupling between the states is also small. 



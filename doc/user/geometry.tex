\label{sec:geom}

\begin{verbatim}
  GEOMETRY [<string name default geometry>] \
           [units <string units default angstrom>] \
           [bqbq] \
           [print [xyz] || noprint] \
           [autoz]
    
    [SYMMETRY [GROUP] <string group_name> [print]]

    geometry specification (cartesian || Z-matrix)

    [ZCOORD ... END]

  END
\end{verbatim}

The input supplied with this directive consists of three main
sections;

\begin{itemize}
\item keywords on the the line of the geometry directive that specify
  the name, units, etc.
\item symmetry information
\item input to specify the locations of the atoms and centers
  (cartesian coordinates or Z-matrix)
\end{itemize}

These are examined in the following subsections.

\subsection{Keywords on the geometry directive}
\subsubsection*{{\tt NAME}}
The default \verb+name+ for a geometry object is \verb+geometry+, and
most modules in the code look for a geometry with this name.  The user
can direct a module to a different geometry by assigning the string
\verb+geomery+ to \verb+name+ using the \verb+SET+ directive (see the
example in Section \ref{sec:set}).

\subsubsection*{{\tt UNITS}}
The default units for the geometry input unit is {\AA}ngstr\"{o}m
(\verb+bohr+ or \verb+au+), however atomic units or Bohr are used
internally.  However, the geometric coordinates can also be supplied
in atomic units, nanometers and picometers by specifying the
appropriate value for the \verb+UNITS+ keyword.  (Note: The default
conversion factor used in the code to convert from {\AA}ngstr\"{o}m to
Bohr is $1.8897265$.)

Possible values for the \verb+UNITS+ keyword are (only the first two
characters need be specified)
\begin{itemize}
  \item \verb+angstrom+ --- {\AA}ngstr\"{o}m, the default.  Converts
   to A.U. using the \AA\ to A.U. conversion factor.
  \item \verb+au+ or \verb+atomic+ or \verb+bohr+ --- Atomic units
  \item \verb+nm+ or \verb+nanometers+ --- nanometers (converts to
     A.U. using a conversion factor computed as $10.0$ times the
     \AA\ to A.U. conversion) 
  \item \verb+pm+ or \verb+picometers+ --- picometers (converts to 
    A.U. using a conversion factor computed as $0.01$ times the 
     \AA\ to A.U. conversion)
\end{itemize}

E.g., the following directives all specify the same geometry for $H_2$
(a bond length of 0.732556\ \AA).
\begin{verbatim}
  geometry
    h 0 0 0
    h 0 0 0.732556
  end

  geometry units nm
    h 0 0 0
    h 0 0 0.0732556
  end

  geometry units pm
    h 0 0 0
    h 0 0 73.2556
  end

  geometry units atomic
    h 0 0 0
    h 0 0 1.3843305
  end
\end{verbatim}
      
\subsubsection*{{\tt BQBQ}}
The default in NWChem is to ignore interactions between dummy centers
(those beginning with \verb+bq+ or \verb+x+),
when computing energies or energy derivatives.  These interactions
will be included if the keyword \verb+BQBQ+ is specified.

\subsubsection*{{\tt PRINT} and {\tt NOPRINT}}
The complimentary keywords \verb+print+ and \verb+noprint+
enable/disable printing of the geometry when it is read by the input
module.  It is printed by default.  In addition, the print keyword may
be qualitifed by the additional keyword \verb+XYZ+ which specifies
that the coordinates should be printed in the XYZ format of XMol.

\subsubsection*{{\tt AUTOZ}}
By default the Cartesian or Z-matrix parameters provided by the user
in the body of the geometry directive are used in subsequent geometry
optimizations.  If \verb+AUTOZ+ is specified then the user's input is
used {\em only} to define the starting geometry and NWChem will
automatically generate a set of internal coordinates suitable for
geometry optimization.  See the Section \ref{sec:zcoord} for how to
force the definition of specific internal variables in combination
with automatically generated variables.

\subsection{Symmetry Group Input}

The symmetry directive is used to specify the point group for the
molecular geometry (space group information for 1-, 2-, and
3-dimensional periodic systems is not yet documented).  The point
group name should be specified as the standard Sch\"{o}files symbol.

The default is no symmetry ($C_1$ point group) and detection of point
group symmetry is not yet automated, though this is planned.  The user
must also know the symmetry of the molecule being modeled, and be able
to specify the coordinates of the symmetry-unique atoms in a suitable
orientation relative to the rotation axes and symmetry planes.
Appendix \ref{symexamples} lists a number of examples of the
\verb+geometry+ directive input for specific molecules having symmetry
patterns recognized by NWChem.

\subsection{Cartesian coordinate input}
\label{sec:cart}

The default in NWChem is to specify the geometry information entirely
in Cartesian coordinates, and examples of this format have already
appeared above (e.g, Section \ref{sec:realsample}). Each center
(usually an atom) is identified on a line of the following form;
\begin{verbatim}

    <string tag> <real x> <real y> <real z> \
        [charge <real charge>] [mass <real mass>]

\end{verbatim}

The string \verb+tag+ is the name of the atom or center and its case
(upper or lower) is important.  The tag is limited to 16 characters
and is interpreted as follows
\begin{itemize}
\item If it begins with either the symbol or name of an element
      (ignoring case) then it is thought to be an atom of that type
      and the default charge is the atomic number adjusted for the
      presence of ECPs (see \ref{sec:ecp}).  Additional characters can
      be used to distinguish between atoms of the same element. E.g.,
      the tags \verb+oxygen+, \verb+O+, \verb+o34+, \verb+olonepair+,
      and \verb+Oxygen-ether+, will all be interpreted as being oxygen
      atoms.  Atoms {\em must} have basis functions associated with
      them (Section \ref{sec:basis}). 
    \item If the tag begins with either \verb+BQ+ or \verb+X+
      (ignoring case) then it is treated as a dummy center with
      default zero charge. Dummy centers may optionally have basis
      functions or non-zero charge.  Note, that in order to recogonize
      the xenon atom (Xe) it is not possible to input a dummy atom 
      beginning with the characters \verb+XE+ --- an attempt to do
      this will generate a xenon atom.
    \item {\em If the tag begins with characters that cannot be
        matched against an atom or \verb+BQ+ or \verb+X+ then a fatal
        error is generated.}
\end{itemize}

It is {\em important} to be aware of the following points
\begin{itemize}
\item The tag of a center is used in the \verb+BASIS+ directive (Section
\ref{sec:basis}) to associate functions with centers.  
\item All centers with the same tag will have the same basis
functions.
\item When automatic symmetry detection is functional only centers
with the same tag will be candidates for testing for symmetry
equivalence.
\item The user specified charges (of all centers, atomic and dummy)
and any net total charge of the system (Section \ref{sec:charge}) are
used to determine the number of electrons.
\end{itemize}

The Cartesian coordinates of the atom in the molecule are specified as real
numbers following the string \verb+tag+.  The user also has the option
of specifying the charge of the atom (or center) and its mass.

The default charge for an atom is its atomic number, adjusted for the
presence of ECPs (see Section \ref{sec:ecp}).  In order to specify a
different value for the charge on a particular atom, the user must
enter the keyword \verb+CHARGE+.

The default mass for an atom is taken to be the mass of its highest
naturally occurring isotope.  If the user wishes to model some other
isotope of the element, its mass must be defined explicitly by
specifying the keyword \verb+MASS+.


\subsection{Z-matrix input}
\label{sec:Z-matrix}

\begin{verbatim}
    (ZMT || ZMATRIX || ZMAT)
      <Z-matrix coordinate specification>
      [VARIABLES
       <list of active variables>]

      [CONSTANTS
       <list of frozen variables>]

    (END || ZEND)
\end{verbatim}

This allows the user to specify the structure of the system with a
Z-matrix which allows specification of the molecular structure using
either cartesian coordinates (see Section \ref{sec:zmcart})or internal
coordinates (bond lengths, bond angles and dihedral angles).  The
Z-matrix input for a center consists of pairs numbers that define
connectivity indices and a bond length and bond or torsion angless.
Cartesian coordinate input consists of three real numbers defining the
x,y,z coordinates of the atom.  {\em Within the Z-matrix input bond
  lengths and cartesian coordinates must presently be specified in
  {\AA}ngstr{\"o}ms, regardless of the entry specified for
  \verb+units+.} Angles are specified in degrees.

%
% When  two  numerical  values,  separated by a comma, are given for some
% variables, they are considered as the initial and final values for the
% definition of a Linearized Synchronous Transit pathway.  The
% geometries generated by linear interpolation between the initial and
% final values.

The centers (denoted as \verb+i+, \verb+j+, and \verb+k+ below) used
to specify Z-matrix connectivity may be given either as integers
(indentifying the centers by number) or as the tag of the center.
{\em If the tag is used, this tag must be unique.} The use of
``dummy'' atoms is possible, by using \verb+X+ or \verb+BQ+ at the
start of the tag.

Bond lengths, bond angles and dihedral angles (denoted below as {\tt
  R}, {\tt alpha}, {\tt beta} respectively) may be specified either as
numerical values or as symbolic strings which are subsequently
defined.  The same sybmolic string may be used several times.  Any
mixture of numeric data and symbols may be given. 

The Z-matrix input is specified sequentially as follows
\begin{verbatim}
   tag1
   tag2 i R
   tag3 i R j alpha
   tag4 i R j alpha k beta [l]
   ...
\end{verbatim}

We examine this in more detail.  In the following, the tag or number
of the center being currently defined is labelled as \verb+C+ (``C''
for current).  Figures \ref{fig:zmat1}, \ref{fig:zmat2} and
\ref{fig:zmat3} display the relationship between the input data 
and the definition of centers and angles.

\begin{figure}[htbp]
\centering
\psfig{figure=zmat1.eps}

\caption{\label{fig:zmat1} Relationship between the centers, bond angle
and dihedral angle in Z-matrix input.}
\end{figure}

\begin{figure}[htbp]
\centering
\psfig{figure=zmat2.eps}

\caption{\label{fig:zmat2} Relationship between the centers and two
  bond angles in Z-matrix input with optional parameter specified as $+1$.}
\end{figure}

\begin{figure}[htbp]
\centering
\psfig{figure=zmat3.eps}

\caption{\label{fig:zmat3} Relationship between the centers and two
  bond angles in Z-matrix input with optional parameter specified as $-1$.}
\end{figure}

\begin{enumerate}

   \item \verb+tag1+

   Only  the  tag  of the first center is required.

   \item \verb+tag2 i R+

     The second center requires specification of its tag and the
     bond-length ($R_{Ci}$) from the first atom which is identified by
     \verb+i+.

   \item \verb+tag3 i R j alpha+

     The third center requires specification of its tag, its distance
     ($R_{Ci}$) to one of the previous two centers (identified by the
     value of \verb+i+) and the angle $\widehat{Cij}$.

   \item \verb+tag i R j alpha k beta [<integer l default 0>]+

     The fourth, and all subsequent centers, require the tag, a bond
     length ($R_{Ci}$) relative to center \verb+i+, the angle with
     centers \verb+i+ and \verb+j+ ($\widehat{Cij}$), and {\em either} 
    \begin{enumerate}
    \item the dihedral angle betwen the current center and centers
      \verb+i+, \verb+j+ and \verb+k+ (Figure \ref{fig:zmat1}), or
      \item  a second bond angle $\widehat{Cik}$ and an orientation to 
      the plane containing the other three centers (Figure
      \ref{fig:zmat2} and \ref{fig:zmat3}).
    \end{enumerate}

    By default $\beta$ is interpreted as a dihedral angle (see Figure
    \ref{fig:zmat1}), but if the optional last parameter (\verb+l+) is
    specified with the value $\pm 1$ then $\beta$ is interpreted as
    the angle $\widehat{Cik}$.  The sign of \verb+l+ specifies the
    direction of the bond angle relative to the plane described by the
    three reference atoms.  If \verb+l+ is $+1$ then the new center
    (\verb+C+) is above the plane (Figures \ref{fig:zmat2}), and if
    \verb+l+ is $-1$ then \verb+C+ is below the plane (Figure
    \ref{fig:zmat3}).
\end{enumerate}

Following the Z-matrix center definitions described above, may be two
optional sections which are described next.

The first must be prefaced by the directive \verb+VARIABLES+ and is
used to define initial values for the symbolic variables used within
the Z-matrix.

\begin{verbatim}
  VARIABLES
    <string symbol>  <double value>
    ...
\end{verbatim}
Each line contains the name of a variable followed by its value.
Optionally, an equals sign (\verb+=+) may be included between the
symbol and its value.

The second section, \verb+CONSTANTS+, is used to define Z-matrix symbolic 
variables that remain unchanged during geometry optimizations.
\begin{verbatim}
  CONSTANTS
    <string symbol>  <double value>
    ...
\end{verbatim}
Each line contains the name of a variable followed by its value.
Optionally, an equals sign (\verb+=+) may be included between the
symbol and its value.
{\em Note that this functionality is not yet available and that all
  Z-matrix parameters (Cartesian coordinates, numerically specified
  bond-lengths and angles, and symbolic variables) will be modified
  during geometry optimizations.}  To freeze the Cartesian coordinates
of an atom refer to Section \ref{sec:activeatoms}).

The end of the Z-matrix input signaled by either \verb+END+ or
\verb+ZEND+.  

A simple example is presented for water.  All Z-matrix parameters are
specified numerically, and the symbolic tags are used to specify
connectivity information.  This requires that all tags are unique, and
so different tags are used for the hydrogen atoms which might 
otherwise be identical.
\begin{verbatim}
  geometry
    zmatrix 
      O
      H1 O 1.08
      H2 O 1.08 H1 108.0
    end
  end
\end{verbatim}

The following example illustrates the Z-matrix input for the molecule
$CH_3CF_3$.  This input uses the numbers of centers when specifying
the connectivity information (\verb+i+, \verb+j+ and \verb+k+) and
uses symbolic variables for the Z-matrix parameters {\tt R}, {\tt
  alpha}, and {\tt beta} which are defined in \verb+VARIABLES+ and
\verb+CONSTANTS+ sections.

\begin{verbatim}
geometry 
 zmatrix
   C 
   C 1 CC 
   H 1 CH1 2 HCH1 
   H 1 CH2 2 HCH2 3  TOR1  0 
   H 1 CH3 2 HCH3 3 -TOR2  0 
   F 2 CF1 1 CCF1 3  TOR3  0 
   F 2 CF2 1 CCF2 6  FCH1  1 
   F 2 CF3 1 CCF3 6  FCH2 -1
   variables
     CC    1.4888 
     CH1   1.0790 
     CH2   1.0789  
     CH3   1.0789  
     CF1   1.3667 
     CF2   1.3669 
     CF3   1.3669
   constants
     HCH1  10428 
     HCH2  10474 
     HCH3  1047 
     CCF1  112.0713 
     CCF2  112.0341 
     CCF3  112.0340 
     TOR1  109.3996 
     TOR2  109.3997 
     TOR3  180.0000 
     FCH1  106.7846 
     FCH2  106.7842
 end   
end
\end{verbatim}

\subsubsection{Using Cartesian variables in Z-matrices}
\label{sec:zmcart}

In order to specify Cartesian coordinates within the Z-matrix it is
necessary to understand the orientation of centers specified using
internal coordinates.  These are arranged as follows:
\begin{itemize}
\item The first center is placed at the origin.
\item The second center is placed along the positive z-axis.
\item The third center is placed in the z-x plane.
\end{itemize}

\subsection{{\tt ZCOORD} --- Defining internal coordinates for {\tt AUTOZ}}
\label{sec:zcoord}

The \verb+AUTOZ+ keyword forces automatic generation of internal
coordinates for use in geometry optimizations.  Connectivity
is inferred by comparing inter-atomic distances with van de Waals
radii, and under some circumstances it may be necessary to augment the
automatically generated list of internal coordinates.  This is
accomplished by including a {\tt ZCOORD} section within the geometry
directive.

The centers \verb+i+, \verb+j+, \verb+k+ and \verb+l+ below {\em must} be
specified using number of the centers.
\begin{verbatim}
   ZCOORD
     ijbond  i j
     ijkang  i j k
     ijklto  i j k l
     ijklop  i j k l
     ijklnb  i j k l
   END
\end{verbatim}

\begin{itemize}
\item {\tt ijbond} --- a bond between the two centers.
\item {\tt ijkang} --- a bond angle $\widehat{ijk}$.
\item {\tt ijklto} --- a torsion (or dihedral) angle.  The
  angle between the planes \verb+i-j-k+ and \verb+j-k-l+.
\item {\tt ijklop} --- an out-of-plane bend.  The angle of center
  \verb+l+ out of the plane \verb+i-j-k+.
\item {\tt ijklnb} --- a linear bend.  This defines two angles
  corresponding to the deformation of the centers \verb+i--j--k+ 
  which may initially be arranged (nearly) linearly.  The center
  \verb+l+ must not be colinear.  The two bends are constructed to be
  within and perpendicular to the plane containing the atoms.
\end{itemize}   

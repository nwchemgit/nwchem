\label{sec:geom}

\begin{verbatim}
  GEOMETRY [<string name geometry>] \
           [units <string units bohr>] \
           [bqbq] \
           [print [xyz] || noprint]
    
    [SYMMETRY GROUP <string group_name> [print]]

    <string tag> <real x> <real y> <real z> [ecp] \
        [charge <real charge>] [mass <real mass>]
    ...
  END
\end{verbatim}

Only Cartesian geometry specification is available, though support for
Z-matrix-like format is proposed.  As mentioned above (section
\ref{sec:arch}), multiple geometries may be stored in the database
provided each is given an independent name.  The default name of
\verb+"geometry"+ is used by most application modules to access the
geometry at which to perform a calculation.  Associating a named
geometry with the required name of \verb+"geometry"+ is described in
section \ref{sec:arch}.  Known names for units are \verb+au+,
\verb+bohr+, or \verb+angstrom+ (the conversion factor used to convert
from Angstr\"{o}m to Bohr is $1.8897265$). By default, the input
module prints any geometry it encounters.  Printing can be disabled
with the \verb+PRINT+ option.  The \verb+XYZ+ qualifier to print
causes the geometry to also be printed in the \verb+XYZ+ format of
XMol.  

Each line in the body of the directive specifies the name or tag, and
the coordinates of one center or atom.  If an effective core potential
is to be used then the \verb+ECP+ keyword should be specified.  The
charge associated with the center is inferred from the atom type
(taking into account defaults for ECPs).  The charge may be explicitly
specified using the \verb+CHARGE+ keyword.  Default masses may be
overriden by specifying the mass.

The tag associated with each center is interpreted as follows:
\begin{itemize}
\item If it begins with \verb+BQ+ (ignoring case) then it is treated
      as a dummy center with default zero charge. Dummy centers may 
      optionally have basis functions or non-zero charge.
\item If it begins with either the symbol or name of an element then
      it is thought to be an atom.  Atoms {\em must} have basis
      functions associated with them and the default charge is the
      atomic number adjusted for the presence of ECPs.  The user
      provided charges (of all centers, atomic and dummy) and the
      total charge of the system are used to determine the number of
      electrons
\item The tag of a center is used in the \verb+BASIS+ directive to
      associate functions with centers.  All centers with the same tag
      will have the same basis functions.  Atomic centers may have
      standard basis sets sited upon them.
\item Similarly, the tag of a center is also used in the \verb+ECP+
      directive to associate ECPs with centers.
\item When automatic symmetry detection is functional only centers
      with the same tag will be candidates for testing for symmetry
      equivalence.
\end{itemize}

By default NWCHEM does not include the interaction between dummy
centers.  The \verb+BQBQ+ qualifier to the \verb+GEOMETRY+ directive
causes these interactions to be included.

 The use of molecular symmetry in NWCHEM is not yet automated, thus,
the user is responsible for detecting symmetry and specifying the
coordinates of the symmetry unique atoms in a suitable orientation
relative to the rotation axes and symmetry planes.  Since it seems
that only the original authors of the symmetry package seem to
understand the latter, we provide many examples in Appendix
\ref{symexamples}.

\label{sec:geom}

The \verb+GEOMETRY+ directive is a compound directive that allows the
user to define the geometry object to be used for a given calculation.
The directive allows the user to specify the geometry 
with a relatively small amount of input, but there are a large number of
optional kewords and additional subordinate directives that the user can
specify, if needed.  The directive therefore appears to be rather
long and complicated when presented in its general form, as follows,
\begin{verbatim}
  GEOMETRY [<string name default geometry>] \
           [units <string units default angstrom>] \
           [bqbq] \
           [print [xyz] || noprint] \
           [autoz]
    
    [SYMMETRY [GROUP] <string group_name> [print]]

    <string tag> <real x y z> [charge <real charge>] [mass <real mass>]
    ... ]

    [ZMATRIX || ZMT || ZMAT
         <string tagn> <list_of_zmatrix_variables> [ghost]
         ... ]

         [VARIABLES
              <string symbol> <double value>
              ... ]
 
         [CONSTANTS
              <string symbol> <double value>
              ... ]

     (END || ZEND)]

         [ZCOORD
              <ij_flag> <list_of_z-coordinate_variables>
               ...
          END]
   END
\end{verbatim}

The three main parts of the \verb+GEOMETRY+ directive
are;

\begin{itemize}
\item keywords on the first line of the directive (to specify such optional
input as the object name, input units, and print level for the output)
\item symmetry information
\item cartesian coordinates or Z-matrix input to specify the locations 
of the atoms and centers
\end{itemize}

The following subsections present the input for this compound directive in
detail, describing the options available and the usages of the various
keywords in each of the three main parts.

% These are examined in the following subsections.

\subsection{Main keywords on the {\tt GEOMETRY} directive}

This section presents the options that can be specified using the keywords 
and optional input on the main line of the {\tt GEOMETRY} directive.
As described above, the first line of the directive has the general form,
\begin{verbatim}
  GEOMETRY [<string name default geometry>] \
           [units <string units default angstrom>] \
           [bqbq] \
           [print [xyz] || noprint] \
           [autoz]
\end{verbatim}
    
All of the keywords and input on this line are optional.  The following
list describes the use and function of each option, including the defaults.

% \subsubsection*{{\tt NAME}}
\begin{itemize}
\item \verb+<name>+ -- user-supplied name for the geometry object; default
name is \verb+geometry+, and
most modules in the code look for a geometry with this name.  The user
can direct a module to a geometry with a different name by assigning
the string \verb+geomery+ to \verb+<name>+ using the \verb+SET+
directive (see the example in Section \ref{sec:set}).

% \subsubsection*{{\tt UNITS}}
\item \verb+units+ -- keyword specifying that a value will be entered
by the user for the string variable \verb+<units>+.  
The default for the geometry input unit is {\AA}ngstr\"{o}m
(Note: atomic units or Bohr are used in the code, regardless of
the option specified for the input units.  
The default
conversion factor used in the code to convert from {\AA}ngstr\"{o}m to
Bohr is $1.8897265$).
The code recognizes the following possible values for the string variable
\verb+<units>+  to specify the input unit for geometry;
\begin{itemize}
  \item \verb+angstrom+ or \verb+an+ --- {\AA}ngstr\"{o}m, the default.  Converts
   to A.U. using the \AA\ to A.U. conversion factor.
  \item \verb+au+ or \verb+atomic+ or \verb+bohr+ --- Atomic units
  \item \verb+nm+ or \verb+nanometers+ --- nanometers (converts to
     A.U. using a conversion factor computed as $10.0$ times the
     \AA\ to A.U. conversion) 
  \item \verb+pm+ or \verb+picometers+ --- picometers (converts to 
    A.U. using a conversion factor computed as $0.01$ times the 
     \AA\ to A.U. conversion)
\end{itemize}
      
% \subsubsection*{{\tt BQBQ}}
\item \verb+bqbq+ -- keyword to specify the treatment of interactions 
between dummy centers.  The default in NWChem is to ignore such interactions
when computing energies or energy derivatives.  These interactions
will be included if the keyword \verb+bqbq+ is specified.

% \subsubsection*{{\tt PRINT} and {\tt NOPRINT}}
\item \verb+print+ and \verb+print+ -- complimentary keyword pair to
enable or disable printing of the geometry.  The default is to print   
the output associated with the geometry.  In addition, the keyword 
\verb+print+ may
be qualifed by the additional keyword \verb+xyz+, which specifies
that the coordinates should be printed in the XYZ format of XMol.

% \subsubsection*{{\tt AUTOZ}}
\item \verb+autoz+ -- keyword to specify that the geometry 
information supplied
with the \verb+GEOMETRY+ directive is to be used {\em only} as the starting
geometry, and NWChem will automatically generate a set of internal coordinates
suitable for geometry optimization.  If this keyword is not specified, then
by default the Cartesian or Z-matrix parameters provided by the user
in the \verb+GEOMETRY+ directive are used in subsequent geometry
optimizations. 
(See Section \ref{sec:zcoord} for a detailed description of how to
use this option, including the way to
force the definition of specific internal variables in combination
with automatically generated variables.)
\end{itemize}

The following examples illustrate some of the various options that the user
can specify on the first input line of the \verb+GEOMETRY+ directive, using
the keywords and input options described above.


The following directives all specify the same geometry for $H_2$
(a bond length of 0.732556\ \AA);
\begin{verbatim}
  geometry
    h 0 0 0
    h 0 0 0.732556
  end

  geometry units nm
    h 0 0 0
    h 0 0 0.0732556
  end

  geometry units pm
    h 0 0 0
    h 0 0 73.2556
  end

  geometry units atomic
    h 0 0 0
    h 0 0 1.3843305
  end
\end{verbatim}
      
\Large
**A few more examples might be useful here.
\normalsize

\subsection{Symmetry Group Input}

The \verb+SYMMETRY+ directive is used within the compound \verb+GEOMETRY+
directive to specify the point group for the
molecular geometry. (Note: space group information for 1-, 2-, and
3-dimensional periodic systems is not yet documented.)
The general form of the directive, as described above within the general
form of the \verb+GEOMETRY+ directive, is as follows,
\begin{verbatim}
    [SYMMETRY [GROUP] <string group_name> [print]]
\end{verbatim}
The keyword \verb+group+ is optional, and can be omitted without affecting
how the input for this directive is processed.
However, a group name must be specified by supplying an entry for the
string variable \verb+<group_name>+.  The
group name should be specified as the standard Sch\"{o}files symbol.
Examples of expected input for the varible \verb+group_name+ include
such entries as,

\begin{itemize}
\item \verb+c2v+ -- for molecular symmetry $C_{2{\it v}}$
\item \verb+d2h+ -- for molecular symmetry $D_{2h}$
\item \verb+Td+ -- for molecular symmetry $T_d$
\item \verb+d6h+ -- for molecular symmetry $D_{6h}$
\end{itemize}

The \verb+SYMMETRY+ directive is optional.  The default is no symmetry 
(i.e., $C_1$ point group), and detection of point
group symmetry is not yet automated, though this is planned.  The user
must know the symmetry of the molecule being modeled, and be able
to specify the coordinates of the symmetry-unique atoms in a suitable
orientation relative to the rotation axes and symmetry planes.
Appendix \ref{symexamples} lists a number of examples of the
\verb+GEOMETRY+ directive input for specific molecules having symmetry
patterns recognized by NWChem.

\subsection{Cartesian coordinate input}
\label{sec:cart}

The default in NWChem is to specify the geometry information entirely
in Cartesian coordinates, and examples of this format have already
appeared above (e.g, Section \ref{sec:realsample}). Each center
(usually an atom) is identified on a line of the following form;
\begin{verbatim}

    <string tag> <real x y z> \
        [charge <real charge>] [mass <real mass>]

\end{verbatim}

The string \verb+<tag>+ is the name of the atom or center, and its case
(upper or lower) is important.  The tag is limited to 16 characters
and is interpreted as follows;
\begin{itemize}
\item If it begins with either the symbol or name of an element
  (regardless of case) then the center is treated as an atom of that type.
  The default charge is the atomic number. 
% (adjusted for the presence of ECPs; see Section \ref{sec:ecp}).  
% this is handled only by the ECP ``nelec'' directive now.  
% RAK
  Additional characters can
  be added to the string, to distinguish between atoms of the same element.
  (For example, the tags \verb+oxygen+, \verb+O+, \verb+o34+, \verb+olonepair+,
  and \verb+Oxygen-ether+, will all be interpreted as oxygen
  atoms.)
\item If the tag begins with the characters \verb+bq+ or \verb+x+
  (regardless of case) then it is treated as a dummy center with
  default zero charge.  (Note: a tag beginning with the characters \verb+xe+
  will be interpreted as a Xenon atom rather than a dummy center.)
  Dummy centers may optionally have basis
  functions or non-zero charge.
  See Section \ref{sec:sample2} for
  a sample input using dummy centers with charges.
\end{itemize}

It is {\em important} to be aware of the following points regarding the
usage and definitions of the values specified for the variable \verb+<tag>+
to describe the centers in a system;
\begin{itemize}
\item If the tag begins with characters that cannot be
    matched against an atom, and those characters are not \verb+BQ+ 
    or \verb+X+, then a fatal
    error is generated.
\item The tag of a center is used in the \verb+BASIS+ directive (Section
\ref{sec:basis}) to associate functions with centers.  
\item All centers with the same tag will have the same basis
functions.
\item When automatic symmetry detection is functional, only centers
with the same tag will be candidates for testing for symmetry
equivalence.
\item The user specified charges (of all centers, atomic and dummy)
and any net total charge of the system (Section \ref{sec:charge}) are
used to determine the number of electrons.
\end{itemize}

\Large
**A few examples would be helpful here.
\normalsize

The Cartesian coordinates of the atom in the molecule are specified as
real numbers supplied for the variables \verb+x+, \verb+y+, and \verb+z+
following the characters entered for the tag.
The values supplied for the coordinates must be in the units specified
by the value of the variable \verb+<units>+ on the first line of
the \verb+GEOMETRY+ directive input.

The cartesian coordinate input line also contains
the optional keywords \verb+charge+ and \verb+mass+, which allow
the user to specify the charge of the atom (or center) and its mass.
The default charge for an atom is its atomic number, adjusted for the
presence of ECPs (see Section \ref{sec:ecp}).  In order to specify a
different value for the charge on a particular atom, the user must
enter the keyword \verb+charge+, followed by the desired charge entered
for the variable \verb+<charge>+.

The default mass for an atom is taken to be the mass of its most abundant
naturally occurring isotope.  If the user wishes to model some other
isotope of the element, its mass must be defined explicitly by
specifying the keyword \verb+mass+, followed by the mass of the isotope
entered for the variable \verb+<mass>+.

The geometry of the system can be entirely specified by Cartesian coordinates,
by supplying a \verb+<tag>+ line of the type described above.  The user has
the option, however, of supplying the geometry of some or all of the atoms 
or centers using a Z-matrix description.  In such a case, the user supplies
the input tag line described above for any centers 
to be described by cartesian coordinates, and then specifies the optional
Z-matrix directive to describe the remainder of the system.

\subsection{Z-matrix input}
\label{sec:Z-matrix}

The \verb+ZMATRIX+ directive is an optional directive that can be used within
the compound \verb+GEOMETRY+ directive to specify the structure of the system with a
Z-matrix.  The Z-matrix option can be used in conjunction with the cartesian
coordinate input described above.  The \verb+ZMATRIX+ directive is a compound
directive that can include the \verb+VARIABLES+ and \verb+CONSANTS+
directives, depending on the options selected.  The general form of the 
compound \verb+ZMATRIX+ 
directive is as follows;
\begin{verbatim}
    [ZMATRIX || ZMT || ZMAT
         <string tagn> <list_of_zmatrix_variables> [ghost]
         ... ]

         [VARIABLES
              <string symbol> <double value>
              ... ]
 
         [CONSTANTS
              <string symbol> <double value>
              ... ]

    (END || ZEND)]
\end{verbatim}

The input module recognized three possible spellings of the directive name,
as noted.  It can be invoked with \verb+ZMATRIX+, \verb+ZMT+, or \verb+ZMAT+.
This compound directive allows the user to specify
the molecular structure using
either cartesian coordinates (see Section \ref{sec:zmcart}) or internal
coordinates (bond lengths, bond angles and dihedral angles).  The
Z-matrix input for a center consists of pairs numbers that define
connectivity indices and a bond length and bond or torsion angles.
Cartesian coordinate input for a center consists of three real numbers defining the
x,y,z coordinates of the atom.  

Within the Z-matrix input, bond lengths and cartesian coordinates must
be input in the user-specified units, as defined by the value specified
for the varible \verb+<units>+ on the first line of the \verb+GEOMETRY+
directive.  All angles are specified in
degrees.

% When  two  numerical  values,  separated by a comma, are given for some
% variables, they are considered as the initial and final values for the
% definition of a Linearized Synchronous Transit pathway.  The
% geometries generated by linear interpolation between the initial and
% final values.

The centers (denoted as \verb+i+, \verb+j+, and \verb+k+ below) used
to specify Z-matrix connectivity may be given either as integers
(indentifying the centers by number) or as the tag of the center.
{\em If the tag is used, this tag must be unique.} The use of
``dummy'' atoms is possible, by using \verb+X+ or \verb+BQ+ at the
start of the tag.

Bond lengths, bond angles and dihedral angles (denoted below as {\tt
  R}, {\tt alpha}, and {\tt beta}, respectively) may be specified either as
numerical values or as symbolic strings that must be subsequently
defined using the input options on the \verb+VARIABLES+ or 
\verb+CONSTANTS+ directives.  The same symbolic string can be used
more than once, and any
mixture of numeric data and symbols is acceptable. Bond angles
($\alpha$) must be in the range $0 < \alpha < 180$.

The Z-matrix input is specified sequentially as follows,
\begin{verbatim}
   tag1
   tag2 i R
   tag3 i R j alpha
   tag4 i R j alpha k beta [orient]
   ...
\end{verbatim}

The structure of this input is examined in more detail below.  
In the following discussion, the tag or number
of the center being currently defined is labelled as \verb+C+ (``C''
for current).  The above discussion on the interpretation of tags
(Section \ref{sec:cart}) also applies to centers defined in Z-matrix
input.  Figures \ref{fig:zmat1}, \ref{fig:zmat2} and \ref{fig:zmat3}
display the relationship between the input data and the definition of
centers and angles.

\begin{figure}[htbp]
\centering
\psfig{figure=zmat1.eps,angle=270,width=6in}

\caption{\label{fig:zmat1} Relationship between the centers, bond angle
and dihedral angle in Z-matrix input.}
\end{figure}

\begin{figure}[htbp]
\centering
\psfig{figure=zmat2.eps,angle=270,width=6in}

\caption{\label{fig:zmat2} Relationship between the centers and two
  bond angles in Z-matrix input with optional parameter specified as $+1$.}
\end{figure}

\begin{figure}[htbp]
\centering
\psfig{figure=zmat3.eps,angle=270,width=6in}

\caption{\label{fig:zmat3} Relationship between the centers and two
  bond angles in Z-matrix input with optional parameter specified as $-1$.}
\end{figure}

\begin{enumerate}

   \item \verb+tag1+

   Only  the  tag  of the first center is required.

   \item \verb+tag2 i R+

     The second center requires specification of its tag and the
     bond-length ($R_{Ci}$) from the first atom which is identified by
     \verb+i+.

   \item \verb+tag3 i R j alpha+

     The third center requires specification of its tag, its distance
     ($R_{Ci}$) to one of the previous two centers (identified by the
     value of \verb+i+) and the angle $\widehat{Cij}$.

   \item \verb+tag i R j alpha k beta [<integer l default 0>]+

     The fourth, and all subsequent centers, require the tag, a bond
     length ($R_{Ci}$) relative to center \verb+i+, the angle with
     centers \verb+i+ and \verb+j+ ($\widehat{Cij}$), and {\em either} 
    \begin{enumerate}
    \item the dihedral angle betwen the current center and centers
      \verb+i+, \verb+j+ and \verb+k+ (Figure \ref{fig:zmat1}), or
      \item  a second bond angle $\widehat{Cik}$ and an orientation to 
      the plane containing the other three centers (Figure
      \ref{fig:zmat2} and \ref{fig:zmat3}).
    \end{enumerate}

    By default $\beta$ is interpreted as a dihedral angle (see Figure
    \ref{fig:zmat1}), but if the optional last parameter (\verb+orient+) is
    specified with the value $\pm 1$ then $\beta$ is interpreted as
    the angle $\widehat{Cik}$.  The sign of \verb+orient+ specifies the
    direction of the bond angle relative to the plane described by the
    three reference atoms.  If \verb+orient+ is $+1$ then the new center
    (\verb+C+) is above the plane (Figures \ref{fig:zmat2}), and if
    \verb+orient+ is $-1$ then \verb+C+ is below the plane (Figure
    \ref{fig:zmat3}).
\end{enumerate}

Following the Z-matrix center definitions described above, the user can
 define initial values for the symbolic variables used in defining the
Z-matrix tags.  This is done using the optional  \verb+VARIABLES+ directive,
which has the general form;

\begin{verbatim}
  VARIABLES
    <string symbol>  <double value>
    ...
\end{verbatim}
Each line contains the name of a variable followed by its value.
Optionally, an equals sign (\verb+=+) may be included between the
symbol and its value, for clarity in reading the input file.

Following the \verb+VARIABLES+ directive, the \verb+CONSTANTS+
directive is used to define Z-matrix symbolic 
variables that remain unchanged during geometry optimizations.  The
general form of this directive is as follows;
\begin{verbatim}
  CONSTANTS
    <string symbol>  <double value>
    ...
\end{verbatim}
Each line contains the name of a variable followed by its value.
Optionally, an equals sign (\verb+=+) may be included between the
symbol and its value.
{\em Note that this functionality is not yet available and that all
  Z-matrix parameters (Cartesian coordinates, numerically specified
  bond-lengths and angles, and symbolic variables) will be modified
  during geometry optimizations.}  To freeze the Cartesian coordinates
of an atom refer to Section \ref{sec:activeatoms}).

The end of the Z-matrix input using the compound \verb+ZMATRIX+
directive signaled by either \verb+END+ or
\verb+ZEND+, following all input for the directive itself and its
associated optional directives.  

A simple example is presented for water.  All Z-matrix parameters are
specified numerically, and the symbolic tags are used to specify
connectivity information.  This requires that all tags are unique, and
so different tags are used for the hydrogen atoms which might 
otherwise be identical.
\begin{verbatim}
  geometry
    zmatrix 
      O
      H1 O 1.08
      H2 O 1.08 H1 108.0
    end
  end
\end{verbatim}

The following example illustrates the Z-matrix input for the molecule
$CH_3CF_3$.  This input uses the numbers of centers when specifying
the connectivity information (\verb+i+, \verb+j+ and \verb+k+) and
uses symbolic variables for the Z-matrix parameters {\tt R}, {\tt
  alpha}, and {\tt beta} which are defined in \verb+VARIABLES+ and
\verb+CONSTANTS+ sections.

\begin{verbatim}
geometry 
 zmatrix
   C 
   C 1 CC 
   H 1 CH1 2 HCH1 
   H 1 CH2 2 HCH2 3  TOR1  0 
   H 1 CH3 2 HCH3 3 -TOR2  0 
   F 2 CF1 1 CCF1 3  TOR3  0 
   F 2 CF2 1 CCF2 6  FCH1  1 
   F 2 CF3 1 CCF3 6  FCH2 -1
   variables
     CC    1.4888 
     CH1   1.0790 
     CH2   1.0789  
     CH3   1.0789  
     CF1   1.3667 
     CF2   1.3669 
     CF3   1.3669
   constants
     HCH1  10428 
     HCH2  10474 
     HCH3  1047 
     CCF1  112.0713 
     CCF2  112.0341 
     CCF3  112.0340 
     TOR1  109.3996 
     TOR2  109.3997 
     TOR3  180.0000 
     FCH1  106.7846 
     FCH2  106.7842
 end   
end
\end{verbatim}

\subsubsection{Using Cartesian variables in Z-matrices}
\label{sec:zmcart}

In order to specify Cartesian coordinates within the Z-matrix it is
necessary to understand the orientation of centers specified using
internal coordinates.  These are arranged as follows:
\begin{itemize}
\item The first center is placed at the origin.
\item The second center is placed along the positive z-axis.
\item The third center is placed in the z-x plane.
\end{itemize}

\subsection{{\tt ZCOORD} --- Defining internal coordinates for {\tt AUTOZ}}
\label{sec:zcoord}

Specifying the keyword \verb+AUTOZ+ on the first line of the \verb+GEOMETRY+
directive forces automatic generation of internal
coordinates for use in geometry optimizations.  Connectivity
is inferred by comparing inter-atomic distances with van de Waals
radii, and under some circumstances it may be necessary to augment the
automatically generated list of internal coordinates.  This is
accomplished by including the optional directive {\tt ZCOORD} within the geometry
directive.  The general form of the \verb+ZCOORD+ directive is as follows,
\begin{verbatim}
         [ZCOORD
              <ij_flag> <list_of_z-coordinate_variables>
               ...
          END]
\end{verbatim}

% The centers \verb+i+, \verb+j+, \verb+k+ and \verb+l+ below {\em must} be
% specified using number of the centers.
The specific input that must be supplied for the \verb+i+, \verb+j+, \verb+k+, and
\verb+l+ is described below.  The centers 
\verb+i+, \verb+j+, \verb+k+ and \verb+l+ are numbered automatically in the
order they are supplied on the input for the \verb+ZMATRIX+ directive.  The
user must adhere to this convention when suppliying the \verb+ZCOORD+
directive input described here.  The bond length and angles for the centers
ae defined as follows;

\begin{verbatim}
   ZCOORD
     ijbond  i j
     ijkang  i j k
     ijklto  i j k l
     ijklop  i j k l
     ijklnb  i j k l
   END
\end{verbatim}

\begin{itemize}
\item {\tt ijbond} --- a bond between the two centers.
\item {\tt ijkang} --- a bond angle $\widehat{ijk}$.
\item {\tt ijklto} --- a torsion (or dihedral) angle.  The
  angle between the planes \verb+i-j-k+ and \verb+j-k-l+.
\item {\tt ijklop} --- an out-of-plane bend.  The angle of center
  \verb+l+ out of the plane \verb+i-j-k+.
\item {\tt ijklnb} --- a linear bend.  This defines two angles
  corresponding to the deformation of the centers \verb+i--j--k+ 
  which may initially be arranged (nearly) linearly.  The center
  \verb+l+ must not be colinear.  The two bends are constructed to be
  within and perpendicular to the plane containing the atoms.
\end{itemize}   

\subsection{Freezing atoms in geometry optimizations}
\label{sec:activeatoms}

Currently the only mechanism for freezing coordinates during a
geometry optimization is to freeze the Cartesian coordinates of a list
of centers.  This is useful for such purposes as optimizing a molecule
absorbed on the surface of a cluster with fixed geometry.  Only the
gradients associated with the active atoms are computed and this can
result in a big computational saving.  Gradients associated with
frozen atoms are forced to zero (note that this destroys certain
translational and rotational invariance properties).

The \verb+SET+ directive (Section \ref{sec:set}) must be used as
follows
\begin{verbatim}
  set geometry:actlist <integer list_of_center_number>
\end{verbatim}
This defines the centers in the list as being active --- all other
centers will have zero force assigned to them and will remain frozen
at their starting coordinates during a geometry optimization.

For instance, the following directive specifies that atom numbers 1,
5, 6, 7, 8, and 15 are active and all other atoms are frozen:
\begin{verbatim}
  set geometry:actlist 1 5:8 15
\end{verbatim}
or equivalently
\begin{verbatim}
  set geometry:actlist 1 5 6 7 8 15
\end{verbatim}


To revert to the default behaviour of all atoms active we
must explicitly delete this entry from the database (since
the database is persistent Section \ref{sec:persist}) as follows
\begin{verbatim}
  unset geometry:actlist
\end{verbatim}





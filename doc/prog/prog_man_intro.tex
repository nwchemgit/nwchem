\section{Purpose}

The purpose of this document is to provide a resource to NWChem
developers that describes the prescribed practices for developing
software that utilize the NWChem source tree.  This document also
delineates the scope, utilization, and features of the base software
development tools (the ``NWChem umbrella'') that are required to
interface with other modules and develop new modules.  

This manual's primary target audience is the NWChem developer.  This
manual assumes that you are somewhat familiar with parallel computing
and computational chemistry.  It also assumes that you know a small
amount about the NWChem software development process.  If you are a
new NWChem developer you should start by reading this manual except
for the detailed application programer interfaces which are provided
as mainly reference.  The ``rule of thumb'' that you should always
follow is if you don't know or understand {\bf ask} someone.  

\section{Organization}

The manual is (or will be) organized with the following chapters:
\begin{itemize}
\item The NWChem architecture. 
\item Memory management and manipulation.
\item How to approach modifying the code (if you dare!).
\item What can be (and maybe should be) in the database.
\item How to get information into and out of the code
\item Brief description of NWChem tools and modules of interest.
\item Appendices with detailed descriptions of available Application
  Programmer Interfaces (APIs).
\end{itemize}


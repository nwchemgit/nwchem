\section{Input Parser}
\label{sec:parser}

The input parser processes the user's input file and translates the
information into a form meaningful to the main program and the driver routines
for specific tasks.  The parser translates input following the 
rules for free-format input specified in the NWChem Users Manual.  The
following subsections present detailed descriptions of the functions
used by the input parser, and the conventional form of the processed
input.

\subsection{Free-format Fortran Input Routines -- INP}

All input routines must be declared in the header file {\tt inp.fh}.

\subsection{Initialization}

\subsubsection{{\tt inp\_init}}

\begin{verbatim}
  subroutine inp_init(ir, iw)
  integer ir, iw     [input]
\end{verbatim}
This function initializes free format input routines to take input from Fortran unit
{\tt ir} and send output to fortran unit {\tt iw}.  The input
file is processed from the current location.

Function {\tt inp\_init()} should be invoked each time the input file is
repositioned using other than {\tt inp\_*()} routines (e.g., rewind).

\subsection{Basic Input Routines}

The basic input routines read the format-free input provided by the user, and 
translate it by the syntax rules defined in the functions.

\subsubsection{{\tt inp\_read}}
\begin{verbatim}
  logical function inp_read()
\end{verbatim}
Read a line from the input and split it into white space (blank or
tab) separated fields.  White space may be incorporated into a field
by enclosing it in quotes (\verb+"+).  The case of input is preserved.
Blank lines are ignored, and text from a pound or hash symbol
(\verb+#+) to the end of the line is treated as a comment.  A
backslash(\verb+\+) at the end of a line (only white space may appear
after it) may be used to concatentate physical input lines into one
logical input line.  A semicolon (\verb+;+) may be used to split a
physical input line into multiple logical input lines.  The special
meaning of hash (\verb+#+), semicolon (\verb+;+) and quotation
(\verb+"+) characters may be avoided only by prefacing them with a
backslash (this must be done even if the character is inside a quoted
character string).

The number of fields read is set to 0, there being a total of
\verb+inp_n_field()+ fields in the line.

If a non-blank line is successfully parsed then \TRUE is returned.
Otherwise an internal error message is set and \FALSE is returned.

Possible errors include detection of EOF ({\tt inp\_eof()} may be used to
check for this condition) or failure to parse the line (e.g., a
character string without a terminating quote).

EOF may be indicated by end of the physical input file, or by a
physical input line that begins with either asterisk (*), period,
or EOF (ignoring case), and has only trailing white space.

The maximum input line width is 1024 characters.

\subsubsection{{\tt inp\_i}}
\begin{verbatim}
  logical function inp_i(i)
  integer i        [output]
\end{verbatim}
Attempt to read the next field as an integer.  Upon success return
\TRUE and advance to the next field.  Otherwise return \FALSE,
save an internal error message and do not change the
current field.  The input argument ({\tt i}) is not changed unless an
integer is successfully read (so that any default value already
present in i is not corrupted).

\subsubsection{{\tt inp\_f}}
\begin{verbatim}
  logical function inp_f(d)
  double precision d       [output]
\end{verbatim}
Attempt to read the next field as a floating point number.  Upon
success return \TRUE and advance to the next field.  Otherwise
return \FALSE, save an internal error message and do not change
the current field.  The input argument ({\tt d}) is not changed unless
an integer is successfully read (so that any default value already
present in {\tt d} is not corrupted).

\subsubsection{{\tt inp\_a}}
\begin{verbatim}
  logical function inp_a(a)
  character *(*) a         [output]
\end{verbatim}
Attempt to read the next field as a character string.  Upon success
return \TRUE and advance to the next field.  Otherwise return
\FALSE, save an internal error message and do not change the
current field.

\subsubsection{{\tt inp\_a\_trunc}}
\begin{verbatim}
  logical function inp_a_trunc(a)
  character *(*) a         [output]
\end{verbatim}
Attempt to read the next field as a character string, quietly
discarding any data that does not fit in the user provided buffer.
Upon success return \TRUE and advance to the next field.
Otherwise return \FALSE, save an internal error message and do
not change the current field.

\subsubsection{{\tt inp\_line}}
\begin{verbatim}
  logical function inp_line(z)
  character*(*) z          [output]
\end{verbatim}
Return in {\tt z} as much of the entire input line as it will hold and
quietly discard any overflow.  Upon success return \TRUE,
otherwise save an internal error message and return \FALSE

\subsubsection{{\tt inp\_cline}}
\begin{verbatim}
  subroutine inp_cline(z, len, success)
  character*(*) z          [output]
  integer len              [input]
  logical success          [input]
\end{verbatim}
A C-callable equivalent of inp\_line, which puts {\tt len -1} characters of
the input line into the character string {\tt z}. Trailing spaces are
eliminated and the string is terminated with a 0 character, as is
standard for C,


\subsubsection{{\tt inp\_irange}}
\begin{verbatim}
  logical function inp_irange(first, last, stride)
  integer first, last, stride     [output]
\end{verbatim}
Attempt to read the next field as a Fortran90-style triplet specifying
a range with optional stride.  Upon success return \TRUE and
advance to the next field.  Otherwise, return \FALSE, save
internal error message, and do not change the current field.  The
input arguments are not changed unless an integer range is
successfully read.

The syntax is \verb+<first>[:<last>[:<stride>]]+, where all terms are
integers.  The default \verb+<stride>+ is 1.  A simple integer is, in
essence, a degenerate triplet, and will be read correctly by this
routine.  The result will be as if the input had been
\verb+"<first>:<first>:1"+.

\subsubsection{{\tt inp\_ilist}}
\begin{verbatim}
  logical function inp_ilist(maxlist, list, n)
  integer maxlist          [input]
  integer list(maxlist)    [output]
  integer n                [output]
\end{verbatim}
Reads the remainder of the line as a list of integers and puts the
results in {\tt list}.  Ranges of integers may be input compactly
using the notation of \verb+inp_irange()+. The number of elements set
from the input is returned in \verb+n+.

\verb+inp_ilist+ returns \TRUE if the input is a valid integer
list, and \FALSE otherwise, also setting an appropriate error
message.  If $n > $ ~{\tt maxlist}, it indicates that there is too
much data on the line to fit in {\tt list}.

\subsubsection{{\tt inp\_search}}
\begin{verbatim}
  logical function inp_search(ocase, z, nz)
  logical ocase            [input]
  integer nz               [input]
  character*(*) z(nz)      [input]
\end{verbatim}
Position the input file at the next logical input line which has a
first input field that matches the leading non-blank characters of one
of the elements of \verb+z+.  If ocase is \TRUE then matches are case
sensitive.

If such a line is found then return \TRUE, and reset the
current input field to 0 (i.e., as if \verb+inp_read()+ had just been
called).

If no such line is found return \FALSE\@.  The file will be
either at EOF or at a line which was not successfully parsed.  EOF may
be detected by \verb+inp_eof()+.

\subsection{Routines concerning fields within a line}

\subsubsection{{\tt inp\_n\_field}}
\begin{verbatim}
  integer function inp_n_field()
\end{verbatim}
Returns the number of fields in the current input line (1, \ldots).  A
value of 0 implies either that EOF or some other error was detected or
{\tt inp\_read()} has not yet been called.

\subsubsection{{\tt inp\_cur\_field}}
\begin{verbatim}
  integer function inp_cur_field()
\end{verbatim}
Returns the number of fields in the input line that have been processed
so far (0, \ldots).  Thus if {\tt inp\_cur\_field()} returns 2, then the next
field read by {\tt inp\_f()} etc.\ will be field 3.

\subsubsection{{\tt inp\_set\_field}}
\begin{verbatim}
  subroutine inp_set_field(value)
  integer value            [input]
\end{verbatim}
Sets the current field (as returned by \verb+inp_cur_field+) to be
value.  $0 \le$~{\tt value}~$\le$ {\tt inp\_n\_field()}.  An out of
range value results in error termination.

\subsubsection{{\tt inp\_prev\_field}}
\begin{verbatim}
  subroutine inp_prev_field()
\end{verbatim}
A convenience routine that positions you to read the field (on the
current input line) that was last read.  It is simply implemented as
\begin{verbatim}
        call inp_set_field(max(0,inp_cur_field()-1))
\end{verbatim}
At the beginning of the line this is a null operation.


\subsection{String routines}
These routines don't actually read input but are helpful in
interpreting input or formatting output.

\subsubsection{{\tt inp\_strlen}}
\begin{verbatim}
  integer function inp_strlen(z)
  character*(*) z          [input]
\end{verbatim}
Return the index of the last non-blank character in {\tt z}, 0 being
returned for a fully blank string.

\subsubsection{{\tt inp\_lcase}}
\begin{verbatim}
  subroutine inp_lcase(z)
  character*(*) z          [input/oputput]
\end{verbatim}
Lowercase the character string {\tt z}.

\subsubsection{{\tt inp\_ucase}}
\begin{verbatim}
  subroutine inp_ucase(z)
  character*(*) z          [input/output]
\end{verbatim}
Uppercase the character string {\tt z}.

\subsubsection{{\tt inp\_compare}}
\begin{verbatim}
  logical function inp_compare(ocase, a, b)
  logical ocase            [input]
  character*(*) a, b       [input]
\end{verbatim}
Return \TRUE iff all the characters in A match the first len(A)
characters of B.  If ocase is \TRUE then comparisons are case
sensitive, otherwise comparisons ignore case.

\subsubsection{{\tt inp\_match}}
\begin{verbatim}
  logical function inp_match(nrec, ocase, test, array, ind)
  integer nrec             [input]
  logical ocase            [input]
  character*(*) test       [input]
  character*(*) array(nrec)[input]
  integer ind              [output]
\end{verbatim}
Let {\tt L} be the length of the character string test ignoring
trailing blanks.  Attempt to find a unique match of \verb+test(1:L)+
against elements of \verb+array(*)+.  If \verb+ocase+ is \TRUE then
comparisons are case sensitive, otherwise comparisons ignore case.

If a unique match is made return the index of the element in
\verb+ind+ and return \TRUE

If the match is ambiguous set \verb+ind+ to 0, and return \FALSE.

If no match is found set \verb+ind+ to -1 and return \FALSE.

\subsubsection{{\tt inp\_strtok}}
\begin{verbatim}
  logical function inp_strtok(z, sep, istart, iend)
  character*(*) z           ! [input] string to parse
  character*(*) sep         ! [input] token separators
  integer istart, iend      ! [output] start/end of next token
\end{verbatim}
Returns the number of the start and end character of the next token in
the character string.  Tokens are separated by one of the characters
in \verb+sep+.  Note that all characters in \verb+sep+ are used including any
trailing blanks.

Before the first call initialize \verb+istart+ to zero, and leave
\verb+istart+ and \verb+iend+ {\em unchanged} for subsequent calls.
Repeated calls return the next token and \TRUE, or \FALSE if there are
no more tokens.  The separators may be changed between calls.  No
internal state is maintained (which is why \verb+istart+ and
\verb+iend+ must not be modified between calls) so multiple strings
may be parsed simultaneously.

E.g., to split the character string \verb+list+ into tokens separated 
by \verb+':'+ and print each token out, you might execute
\begin{verbatim}
     istart = 0
  10 if (inp_strtok(list, ':', istart, iend)) then
        write(6,*) list(istart:iend)
        goto 10
     endif
\end{verbatim}

\subsection{Error handling routines}

\subsubsection{{\tt inp\_errout}}
\begin{verbatim}
  subroutine inp_errout()
\end{verbatim}
If there is an internal error message, print out its value, the
current line number and its contents.  If appropriate indicate the
problematic position in the current input line.

\subsubsection{{\tt inp\_outrec}}
\begin{verbatim}
  subroutine inp_outrec()
\end{verbatim}
Print out the current input line.

\subsubsection{{\tt inp\_clear\_err}}
\begin{verbatim}
  subroutine inp_clear_err()
\end{verbatim}
Clear error conditions and messages that may no longer be relevant.
For instance, if values are read from a line until no more are
available, the error message ``at end of line looking for \ldots''
will be internally recorded.  A call to this routine will clear this state.

\subsubsection{{\tt inp\_eof}}
\begin{verbatim}
  logical function inp_eof()
\end{verbatim}
Return \TRUE if EOF has been detected, \FALSE otherwise.



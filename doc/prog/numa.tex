% new section to introduce MA, GA, and ChemIO as the menage a trios known as NUMA

\label{sec:numa}


In order to scale to massively parallel computer 
architectures in all aspects of the hardware: CPU, disk,
and memory, NWChem uses the principle of Non-Uniform Memory Access (NUMA)
to distribute the data across all nodes.
The MA tool allows the programmer to allocate memory that is local to
the calling process.  This is data that will generally not be directly
shared with other processes, such as workspace for a particular local
calculation or for replication of very small sets of data.

The GA tool supports the NUMA model by allowing nodes to share arrays between
processes as if the memory is physically shared.  It allows the
programmer to use relatively simple routines to access and manipulate
data in the shared arrays.  The trade-off with this approach is that
access to shared data will be slower than access
to local data, and the programmer must be aware of this in designing modules.

The following subsections describe the Memory Allocator library and 
the Global Arrays library and how they are used in NWChem.


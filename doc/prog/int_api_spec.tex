%
% $Id: int_api_spec.tex,v 1.1 1997-05-28 08:23:05 d3e129 Exp $
%
%-----------------------------------------------------------------%  
%                                                                 % 
%                                                                 % 
%  This latex source file should not be edited.  It is generated  % 
%  automatically from the integral API source code using "seetex" % 
%  modifications required sould be made in the source code which  % 
%  is in the source directory ".../nwchem/src/NWints/api" from    % 
%  the standard repository. If you have questions or problems     % 
%  contact Ricky Kendall at ra_kendall@pnl.gov or (509)375-2602   % 
%                                                                 % 
%                                                                 % 
%-----------------------------------------------------------------%  
\chapter{Integral Application Programmer's Interface} 
This appendix describes the interface to all routines for the NWChem 
integral API.  
\section{INT-API: Initialization, Integral Accuracy and Termination} 
These routines set the scope for the integral computation that is 
about to be performed.  
% 
 
%API Initialization and Termination Routines 
\subsection{int\_init} 
This is the main initialization routine for integrals. 
Default memory requirements, accuracy thresholds, and other  
initializations for all base integral codes are set here.  
This routine will read (from the rtdb) any integral  
settings changed by the user. 
 
{\it Syntax:} 
\begin{verbatim} 
      subroutine int_init(rtdb, nbas, bases) 
\end{verbatim} 
\begin{verbatim} 
      integer rtdb        ! [input] run time data base handle 
      integer nbas        ! [input] number of basis sets to be used 
      integer bases(nbas) ! [input] basis set handles 
\end{verbatim} 
%API Initialization and Termination Routines 
\subsection{intd\_init} 
This is the main initialization routine for integral derivatives. 
Default memory requirements, accuracy thresholds, and other  
initializations for all base integral codes are set here.  
This routine will read (from the rtdb) any integral  
settings changed by the user. 
 
{\it Syntax:} 
\begin{verbatim} 
      subroutine intd_init(rtdb,nbas,bases) 
\end{verbatim} 
\begin{verbatim} 
      integer rtdb        ! [input] run time data base handle 
      integer nbas        ! [input] number of basis sets to be used 
      integer bases(nbas) ! [input] basis set handles 
\end{verbatim} 
 
%API Initialization and Termination Routines 
\subsection{int\_terminate} 
This is the main termination routine for integrals. 
After this call the INT-API is ready for re-initialization. 
 
{\it Syntax:} 
\begin{verbatim} 
      subroutine int_terminate() 
\end{verbatim} 
No formal arguments 
 
 
%API Initialization and Termination Routines 
\subsection{intd\_terminate} 
This is the main termination routine for integral 
derivatives. 
After this call the INT-API is ready for re-initialization. 
 
{\it Syntax:} 
\begin{verbatim} 
      subroutine intd_terminate() 
\end{verbatim} 
No formal arguments 
 
% part of API Internal Routines 
\subsection{int\_acc\_std} 
This routine sets the integral threshold for radial cutoffs in all  
integral codes used in the api via a parameter statement.  Other 
routines have access via the apiP.fh common blocks and the set/get API. 
 
{\it Syntax:} 
\begin{verbatim} 
      subroutine int_acc_std() 
\end{verbatim} 
\begin{verbatim} 
      parameter(val_def = 1.0d-10) 
\end{verbatim} 
% part of API Internal Routines 
\subsection{int\_acc\_high} 
This routine sets the integral threshold to ``high'' accuracy  
for radial cutoffs in all integral codes used in the api via a  
parameter statement.  Other routines have access via the apiP.fh  
common blocks and the set/get API. 
 
{\it Syntax:} 
\begin{verbatim} 
      subroutine int_acc_high() 
\end{verbatim} 
\begin{verbatim} 
      val_int_acc = 1.0d-20 
\end{verbatim} 
% part of API Internal Routines 
\subsection{int\_acc\_get} 
This routine returns the current integral threshold  
for radial cutoffs in all integral codes used in the api via a  
parameter statement.   
 
{\it Syntax:} 
\begin{verbatim} 
      subroutine int_acc_get(retval) 
\end{verbatim} 
\begin{verbatim} 
      double precision retval ! [output] current threshold 
\end{verbatim} 
% part of API Internal Routines 
\subsection{int\_acc\_set} 
This routine sets the current integral threshold  
for radial cutoffs in all integral codes used in the api via a  
parameter statement.   
 
{\it Syntax:} 
\begin{verbatim} 
      subroutine int_acc_set(setval) 
\end{verbatim} 
\begin{verbatim} 
      double precision setval ! [input] new threshold 
\end{verbatim} 

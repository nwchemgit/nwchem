\section{Coding Style}
\label{sec:coding-style}
%%%%%%%%%%%%%%%%%%%%%%%%%%%%%%%%%%%%%%%%%%%%%%%%%%%%%%%%%%%%%%%%%%%%%%%%%%%%%%%

In a project this large, it is necessary to impose some standards on
the coding style employed by developers.  The primary goal of these
standards is not to constrain developers, but to enhance both the
quality of the final product and its functionality. 

Code quality is somewhat subjective, but clearly embraces the ideas of
\begin{itemize}
\item correctness 
\item maintainability 
\item efficiency
\item readability 
\item re-usability
\item modularity 
\item ease of integration with other packages
\item speed of development
\item density of bugs
\item ease of debugging
\item detection of errors at run time
\item exposure of available functionality
\item ease-of-use of the API
\end{itemize}
Compromise is clearly necessary. We are interested in
high-performance, so some key kernels may sacrifice readability (but
perhaps not modularity) for efficiency, but most code (i.e., 99.9\%)
is not an inner loop in need of such optimization, as long as the
overall structure is correct.

The single most important thing you can do to achieve quality code has
little to do with programming style. It is the {\em design} --- putting
in the necessary thought and effort before even a single line of code
is written. 

The following subsections present the recommended "Do's and Don'ts"
for programming modules and modifications in NWChem.
The recommendations are organized by a 'top-down' logic, to reflect the
most efficient order in thinking about the various considerations
the developer must keep in mind when designing a new piece of code.


\subsection{Version information}

Each source file should include a comment line that contains the CVS
revision and date information.  This is accomplished by including a
comment line containing the string \verb+$+\verb+Id+\verb+$+.  CVS
substitutes the correct version information each time the file is
checked out or updated.  These lines are processed from the source and can be
output at runtime to aid in bug-tracking.

\subsection{Standard interface for top-level modules}

In order to allow for automatic configuration of various modules in
a compilation of NWChem (to control the size of the executable
in memory-critical situations), all top-level modules must have a
standard interface.  Currently it looks like this;
\begin{verbatim}
  logical function MODULE(rtdb)
\end{verbatim}
The argument \verb+rtdb+ is the handle for the run-time database.  The
function should return \TRUE\ or \FALSE\ on success or failure
respectively.

The only sources of information for a module are the database, or files
with names that can be inferred from data in the database or from defaults.
Futhermore the naming of database entries is standardized such that:
\begin{itemize}
\item The string with which database entries are prefixed must be
  lowercase and match the module name used in the input.  E.g., input
  for the SCF module appears in the \verb+scf;...;end+ block and the
  prefix used in the databse is \verb+scf+.  This is so that the user
  can delete all state information using the \verb+UNSET+ directive.
\item Common quantities (such as energy, gradient, \ldots) should be
  stored using that name.  E.g., \verb+scf:energy+.
\end{itemize}


\subsection{No globally defined common blocks}

Use of global variables (e.g., common blocks) is generally a bad idea.
Such variables break modularity, form hidden dependencies and make
code hard to reuse and maintain.  {\em Do not use common blocks to
  pass data between routines}.

However, common blocks are very useful in supporting a modular
programming style which encourages code reuse and improves
maintainability.  To this end common blocks can be used to hide data
behind a subroutine interface so that access to the common is limited
to a few tightly integrated routines.  The benefits of using common
blocks (smaller argument lists, static data allocation, contiguous
memory layout) can thus, with care, be realized without any problems.
Examples of this include the basis, geometry, RTDB, integral,
symmetry, global array, message passing, SCF, optimizer, input, and
MP2 libraries.

\subsection{Naming of routines and common blocks}

To avoid name clashes and for easy identification, prefix all
subroutine, function and common block names with the name of the
module they are associated with.  For instance,
\begin{itemize}
\item {\tt rtdb\_\ldots} --- run-time database
\item {\tt ma\_\ldots} --- memory allocator
\item {\tt ga\_\ldots} --- global array
\item {\tt scf\_\ldots} --- SCF
\item {\tt stpr\_\ldots} --- Stepper (geometry optimization)
\end{itemize}


\subsection{Inclusion of common block definitions}

All common block definitions, including typing of variables in the
common, are to be made once only in a single file (a {\tt.fh} file),
that is included in other source using the C preprocessor.  The
\verb+include+ file should document the meaning of all variables.  
This helps ensure that variables in a common block are consistently named and
dependencies of routines on common blocks are easily generated
and maintained.

\subsection{Convention for naming {\tt include} files}

All \verb+include+ files should be named using the following conventions,
\begin{itemize}
\item Use \verb+.fh+ for files that can be included only by Fortran routines
\item Use \verb+.h+ for files that can be included by C routines only, or
  for files that are included by both C and Fortran routines
\end{itemize}

\subsection{Syntax for including files using the C preprocessor}

A very important distinction hinges on the seemingly trivial difference between
the two different \verb+include+ forms,
\begin{itemize}
\item \verb+#include "filename"+
\item \verb+#include <filename>+
\end{itemize}
According to Kernighan and Ritchie:
\begin{quotation}
 "If the {\em filename} is quoted, searching for the file typically
 begins where the source program was found; if it is not found there,
 or if the name is enclosed in \verb+<+ and \verb+>+, searching follows
 an implementation-defined rule to find the file."
\end{quotation}
For this reason, and by common convention, only system-defined \verb+include+
files are included using angle brackets.  Those \verb+include+ files that are defined
within an application are included using quotes.  The
automatic generation of dependencies of source files upon \verb+include+
files within NWChem {\em relies} upon this convention.


\subsection{No implicitly typed variables}

The command {\tt implicit none} should appear at the top of every routine
in the NWChem code.  No other
implicit statements are permitted and all variables must be explicitly
declared.  {\em This rule should be religiously observed in new code.} It
\begin{itemize}
\item lets the compiler help you find typos and other errors
\item makes the code more readable and more maintainable
\item provides a natural point to document arguments and local
  variables
\item makes silly variable names like {\tt iii, ii1} both
  obvious and even more embarrassing when others catch you doing it
\end{itemize}

When integrating existing code, this rule may seem to be more work than
it is worth, but several bugs in existing code have been found in this
fashion.

\subsection{Use {\tt double precision} rather than {\tt real*8}}

{\tt REAL*8} is not standard Fortran.  {\tt DOUBLE PRECISION} is the
standard, it is usually what you want, it is more portable, and
standardization of declarations enables us to perform 
necessary code transformations more readily.

\subsection{C macro definitions should be in upper case}

NWChem uses the ANSI C preprocessor to handle machine dependencies and
other conditional compilation requirements.  By forcing all C macros
to be upper case the code is made more readable and we also avoid
potential accidental munging of Fortran source.  This practice is
consistent with conventional use of the preprocessor in C programs.

\subsection{Fortran source should be in lower or mixed case}

This convention is complementary to the above C macro convention.
If there are no fully upper-case Fortran tokens then there can
be no accidental conflict with the C preprocessor.


\subsection{Naming of variables holding handles/pointers obtained from
  MA/GA}

So that these critical variables are immediately recognizable, the
following conventions are recommended.
\begin{itemize}
\item handles obtained from MA should be prefaced with {\tt l\_}
\item pointers (into {\tt dbl\_mb()}, etc.) obtained from MA should be
  prefaced with {\tt k\_}
\item handles obtained from GA should be prefaced with {\tt g\_}
\end{itemize}

Alternatively, you can insert 
comment lines describing the
variables at the point of declaration, if you do not want to
follow these conventions.

\subsection{Fortran unit numbers}

All references to Fortran I/O units should be done with parameters or
variables instead of hardwired constants.  For the ``standard I/O''
units, corresponding to the C \verb+stdin+, \verb+stdout+, and \verb+stderr+, you should
include the file {\tt stdio.fh} and use the variables \verb+luin+,
\verb+luout+, and \verb+luerr+ instead of 5, 6, and 0.

The code uses very few other files, and there is no organized list 
of parameter names for
non-standard I/O units.  Users are free to use parameter names that make sense
to them, so long as they adhere to the convention.  Using parameters rather
than hardwired integer constants helps insure that I/O unit designations
can be changed easily if needed, and may facilitate moving to a
more general convention in a future version of the code.

\subsection{Use standard print control}

All modules should understand the \verb+PRINT+ directive and
accept at least the following keywords for this
\begin{itemize}
\item \verb+none+ --- no output whatsoever except for error messages
\item \verb+low+ --- minimal output; e.g., title, critical parameters
and a final energy
\item \verb+medium+ = \verb+default+ --- usual output
\item \verb+high+ --- extra verbose output
\item \verb+debug+ --- anything useful for diagnosing problems
\end{itemize}

Ideally all applications should control most printing via the print
control routines (see Section \ref{sec:print}).  A uniform
look and feel is important.

\subsection{Error handling}

All fatal errors should result in a call to \verb+errquit()+ (see Section
\ref{errquit}), which prints out the string and status value to both
standard error and standard output and attempts to kill all parallel
processes and to tidy any allocated system resources (e.g., system V
shared memory).

\subsection{Comments}

The use of comment lines is strongly recommended in all coding.
Commented code is easier to read, and often is easier to debug, maintain,
and modify.  Liberal use of comments is particularly important in NWChem, 
since it is used by a large and diverse group of people, it is constantly being
modified as capabilities are added and refined, and it has only a limited
amount of detailed documentation.

Requirements for in-source documentation are given in detail in 
Chapter \ref{sec:newdoc}
but the general recommendation for comment lines in the code is
the more the merrier.  At a minimum the source code should
be able to provide the following information, 
\begin{itemize}
\item terse comments at the top of each subroutine to describe
  (accurately!) its function,
\item documentation of dependencies/effects on state that are not passed
  directly through its argument list (e.g., files, the database, common
  blocks)
\item descriptions of all arguments, including the flow of information (i.e.,
  label arguments as input or output, or input-output)
\item documentation of local variables with functions that are not apparent
  from their names, or which have an algorithmic role that is opaque or obscure
\end{itemize}

In some circumstances, comments at the top of a routine can be quite
lengthy since this is a {\tt very} good place to store details of the
algorithm.
% When an interface is finalized and is to be exported for use by others,
% it should be documented here in {\tt prog.tex} --- nearly all of the
% documentation in this file was generated by pasting in existing 
% comments in the source.  
Automatic generation of documentation
from code comments is being designed, 
but this will produce useful documentation
{\em only if
developers write clear and concise commentary in the code as they work.}

The following partial listings show examples of minimalist in-source documentation
using comment lines.  It would not be difficult to say more.  The rule of 
thumb should be "from those who have much, more will be expected". The more important
a routine is to a particular algorithm, the more it does in the way of carrying
out the solution, the more detailed and voluminous should be it's comment lines.

Example of comments in a simple routine:

\begin{verbatim}
  logical function bas_numbf(basis,nbf)
  implicit none
  integer basis   ! [input] basis set handle         
  integer nbf     ! [output] number of basis functions
*
*  nbf returns the total number of functions.
*  Returns true on success, false if the handle is invalid
*  
\end{verbatim}

Example of comments in a less simple routine:

\begin{verbatim}
      subroutine sym_symmetrize(geom, basis, odensity, g_a)
C$Id$
      implicit none
      integer geom, basis  ! [input] Handles
      integer g_a          ! [input] Handle to input/output GA
      logical odensity     ! [input] True if matrix is a density
c
c     Symmetrize a skeleton matrix (in a global array) in the
c     given basis set.
c
c     A <- (1/2h) * sum(R) [RT * (A + AT) * R]
c
c     where h = the order of the group and R = operators of the
c     group (including the identity)
c
c     Note that density matrices transform according to slightly
c     different rules to Hamiltonian matrices if components
c     of a shell (e.g., Cartesian d's) are not orthonormal.
c     (see Dupuis and King, IJQC 11, 613-625, 1977)
\end{verbatim}

\subsection{Message IDs}

The use of tags/IDs/types on messages is strongly suggested.
If all messages with the program
have distinct types and the message-passing software forces the types
of messages to match between sender and receiver, then
there is a way to prove that messages
are being sent and received correctly.  If they are not, a 
runtime error will be detected.
This is especially important
to NWChem since the code makes use of many third party linear algebra libraries that
do a lot of message passing.

Modules which do a significant amount of messaging 
should reserve a section of the message
ID space for their own use (e.g., GA or PEIGS).  Most modules, however, do
only a small amount of messaging.  For these, the \verb+include+ file {\tt
  msgids.fh} should be used to reserve individual message IDs.  This
file defines Fortran parameters for message IDs used in most NWChem
{\em Hardwired message IDs should not be used in any NWChem routine.}


\subsection{Bit operations --- {\tt bitops.fh}}

The following bitwise operations (see Table
\ref{tabbit} for definitions) are the recommended standards for use in
NWChem.
\begin{itemize}
\item \verb+ior(i,j)+ --- inclusive OR
\item \verb+ieor(i,j)+ --- exlusive OR
\item \verb+iand(i,j)+ --- AND
\item \verb+not(i)+ --- NOT or one's complement
\item \verb+rshift(i,nbits)+ --- right shift with zero fill
\item \verb+lshift(i,nbits)+ --- left shift with zero fill
\end{itemize}
\begin{table}[h]
\begin{center}

\begin{tabular}{c|c|c|c|c|c}
 ior   &   ieor   &    iand  &   not  & lshift   & rshift   \\ \hline
       &          &          &        &          &          \\
 110   &   110    &    110   &   10   & 10111011 & 10111011 \\
 100   &   100    &    100   &        & 2 bits   & 2 bits   \\ \hline
 110   &   010    &    100   &   01   & 11101100 & 00101110 
\end{tabular}
\vspace{0.2in}
\caption{\label{tabbit} Effect of Bit Operations}


\end{center}
\end{table}

These operations are readily generated using
in-line functions from most other definitions.
The shift examples in Table 3.1
use an eight bit word written with the most significant bit on the
left.
All operations operate on full integer words (32 or 64 bit as
necessary) and produce integer results.  The declarations and any
necessary statement functions are in \verb+bitops.fh+.  The presence
of data statements makes it impossible to have a single include file
make declarations and define statement functions.  To circumvent this
the declarations are in \verb+bitops_decls.fh+ and the statement
functions are in \verb+bitops_funcs.fh+.

\subsection{Blockdata statements and linking}

At least one machine (the CRAY-T3D) discards all symbols that are not
explicitly referenced, even if other symbols from the same \verb+.o+
file are used.  Thus, \verb+BLOCK DATA+ subprograms are not linked in.
One fix to this is to declare each \verb+BLOCK DATA+ subprogram as an
undefined external on the link command, but this makes the link
command depend on the list of modules being built.  An alternative
mechanism that works on the T3D is to reference each \verb+BLOCK DATA+
subprogram in an \verb+EXTERNAL+ statement within a \verb+SUBROUTINE+ or
\verb+FUNCTION+ that is guaranteed to be linked if any reference is to
be made to the \verb+COMMON+ block being initialized.



\label{sec:newdoc}


{\em "Yes, of course we must document the code."}  --rjh

In keeping with the top-down approach outlined in Chapter \ref{sec:develop}
for developing new modules or enhancements to the code, the general
approach to documentation should also reflect forethought and planning.
The purpose of documentation is not only to communicate clearly and efficiently
to new developers the existing structure of the code, but also to define the desired
structure and organization of new code.  Activities that require documentation
fall into three broad categories;

\begin{enumerate}
\item development of a new molecular calculation module
\item modification or enhancement of an existing molecular calculation module
\item modification or enhancement of elements of any of the toolkits 
(i.e., the molecular modeling
tools, the software development toolkit, or the utilities)
\end{enumerate}

The following subsections present guidelines and templates 
to help the developer produce useful documentation for work in each of
these categories.  The basic philosophy of documentation in NWChem 
is to have as much of the documentation as possible in the source code itself.
This is where one most likely would be looking when most in
need of guidance.  This approach also holds forth the shining hope that
one day we will develop a system that allows the in-source documentation
to be automatically extracted for inclusion in updated versions of
this manual.  However, for the high level modules in the code, the level
of detail required for the documentation to be useful will generally
result in too much verbage to be readily included in the the source code as
comment lines.  So some
additional documentation will always be necessary.

\section{Documentation of a New Molecular Calculation Module}
\label{sec:standalone}

The documentation of a new molecular calculation module for NWChem will generally
require creating a stand-alone Latex document.  This document should reside
in the directory containing the source code for the module (or in a subdirectory
named {\tt doc} within that directory).  It is the responsibility of the
developer to write the documentation as an integral part of the development
process, and then to keep it current as changes or modifications are made
in the module.  The developer can, as an alternative to writing a seperate
LaTex document, put the documentation directly in the main subroutine of
the module.  Whichever approach is used, however, the documentation should
conform to the following template.

\subsection{Template for Documentation of a Molecular Calculation Module}

This template shows the required format for documentation of a molecular
calculation module for NWChem.  This template can be used for in-source
documentation, or in a stand-alone Latex document.  If the developer chooses to write
a stand-alone LaTex document, the file should reside in the same directory as
the source code, and the name of the file should be given the suffix
{\tt .doc}.  Chapter \ref{sec:modules} contains documentation of existing
molecular calculation modules in the code, and can be refered to for guidance
on style and appropriate level of detail.  (However, developers are encouraged
to write in their own unique style, so long as the necessary information is
communicated in a clear and concise manner.  {\em "A foolish consistency is the bugaboo 
of small minds."})

\subsubsection{Module Documentation Template}

\begin{itemize}

\item Introduction

\begin{itemize}
\item Give a brief, but {\em non-recursive} description of the module, noting
whatever might be unique about it, or what makes it worth the trouble of
adding it to NWChem.
\item Note source of the underlying work, listing collaborators, if any, with full
bibliography (if available); note any significant geneological information, if relevent.
\end{itemize}

\item Overview

\begin{itemize}
\item Describe the theory used by the module, and the operation(s) it performs.
\item Describe how the module interacts with the rest of
NWChem.
\end{itemize}

\item Solution Procedure

\begin{itemize}
\item Describe the numerical solution used by the module, including how it interfaces
with the NUMA model for memory management used by NWChem.
\item If the new module calls other modules in the code, describe in detail how
this interface occurs (e.g., what does it write to the runtime data base?).
\end{itemize}

\item Performance Evaluation

\begin{itemize}
\item Describe the testing of this module, and it's performance as evaluated by
the criteria defined in Chapter \ref{sec:testing}.
\item Present results of applications showing the capability of the module, and
where possible, comparing to results of other modules.  (This may be for validation
as well as evaluation.)
\end{itemize}

\end{itemize}

When modifying or enhancing an existing module, the documentation should also be
updated to match the new form of the module.  If by chance the module does not
yet have adequate documentation, 
this is an opportunity for you to gain merit (in this world
{\em and} the next) by providing the missing information as well as supplying the
documentation for the new coding.

\section{Documentation of New Modeling or Development Tools}

In-source documentation generally makes the most sense for new modeling tools
or development tools, but the developer is free to write as much as he or she
wishes.  Stand-alone LaTex documents will not be spurned.  However, the
in-source documentation should also be done for modules at this level, since
this is perhaps the only way to stack the odds in favor of continual updating
of the documentation as the coding is changed.  The following subsections
present a template for documentation of new modeling tools, and the
required syntax for in-source documentation on the level of subroutines and
functions.

\subsection{Template for Documentation of New Modeling Tools}

This template is based on the structure of the in-source documentation
developed by Ricky Kendall for the Integral API.  The format
is general enough, however, to be applicable to almost any 
feature that might be added to the Molecular Modeling Toolkit or
the Software Development Toolkit.  The template consists of four main parts; 
an introduction, an overview, special instructions regarding modifications
or enhancements to the feature, and a detailed description of all of
its subroutines
and functions.  


Documentation of a code or module can be thought of as a dialogue between the
developer and future developers or users of the code in which the original developer
must guess the questions the other person will ask.  Fortunately, this is 
not all that difficult.  If the documentation is written to answer these 
questions, then it is quite likely that the next person to pick up the
code will readily understand what it is supposed to do, how it works, and
may even be able to figure out how to fix it when it is broken.  The template
described below, therefore, is presented in terms of the questions each
section of the documentation should be written to answer.

\subsubsection{Modeling Tools Documentation Template}

\begin{itemize}

\item Introduction

\begin{itemize}
\item What is this thing? (List it's name and a brief--but {\em non-recursive} --
description of what it does.)
\item Where did it come from?  (List source references, if any, with full
bibliography (if available); note any significant geneological information, if relevent.)
\end{itemize}

\item Overview

\begin{itemize}
\item What does it do? (Give a detailed, nuts and bolts description of what
the code does, and how it does it.  Describe how it interacts with the rest of
NWChem.  If there are any special requirements or limitations on the use of the
feature(s) of this coding, this is the place to
mention them.
\end{itemize}

\item Modifications

\begin{itemize}
\item Can this coding be changed?  (Describe any special considerations for
modifying the code, especially if there are hidden repercussions of choices made
at this level in the code.  Note any compatibility problems with other modules
in the code.)
\end{itemize}

\item Annotated List of All Subroutines and Functions

\begin{itemize}
\item How many subroutines/functions are there in this element?  (Note the number; if
it is large, try to organize them into some sort of logical groupings, for ease of
reference and to clarify the structure of the coding.  If there is no obvious structure,
present them in alphabetical order.)
\item What are the subroutines/functions in this element?  (For each subroutine
or function,

\begin{itemize}
\item write a terse description of what it is for and what it does
\item reproduce the header line verbatim
\item construct a list of all arguments in the call list, and give
each argument with a concise (but informative) definition
\item identify each argument as input or output
\item note the return value(s) of the function itself, in addition to
the argument list
\end{itemize}

\end{itemize}

\end{itemize}

\subsection{Syntax for In-Source Documentation of Routines}

This is the base level of documentation, and is the one level that is almost
guaranteed to actually be read by a new developer.  Therefore, it is very
important that the documentation at this level be as clear and complete as
possible.  At the very minimum, the in-source documentation should consist of
comment lines
containing
the following information:

\begin{itemize}
\item a verbatim reproduction of the routine header line
\item a list of all arguments, identifying for each argument
\begin{itemize}
\item its data type
\item its status as input or output data
\item a concise (but informative) definition
\end{itemize}
\item a terse description of what the routine does
\item a description of the return value(s) of the function itself (if any)
\item a description of action on detecting an error condition
\item a terse description of input and output parameters the function
gets from or gives to an API
\end{itemize}

Examples of nicely documented routines can be found in some directories of
the NWChem source tree.  (There are also many poor examples, so please follow
the above template and do not rely
on the form of existing code for guidance.)  Some examples are reproduced
here, to illustrate good to adequate documentation.  (There are no really
outstandingly excellent examples in the code, as yet.  Think of it as your
opportunity to shine.)

Example 1: in-source documentation of function {\tt rtdb\_parallel}


\begin{verbatim}
  logical function rtdb_parallel(mode)
  logical mode              [input]
\end{verbatim}
This function sets the parallel access mode of all databases to {\tt mode} and returns the
previous setting. If {\tt mode} is true then accesses are in parallel, otherwise they are
sequential.

{\em Comment:} This function comes close to meeting the requirements of the
desired level of documentation.  It is lacking only a definition of the argument {\tt mode},
and it could be argued that in this case the definition is obvious.


Example 2: in-source documentation of function {\tt task\_energy}


\begin{verbatim}
      logical function task_energy(rtdb)
      integer rtdb
c
c     RTDB input parameters
c     ---------------------
c     task:theory (string) - name of (QM) level of theory to use
c     
c     RTDB output parameters
c     ----------------------
c     task:status (logical)- T/F for success/failure
c     if (status) then
c     .  task:energy (real)   - total energy
c     .  task:dipole(real(3)) - total dipole moment if available
c     .  task:cputime (real)  - cpu time to execute the task
c     .  task:walltime (real) - wall time to execute the task
c
c     Also returns status through the function value
c
\end{verbatim}


{\em Comment:} This is also a fairly good example.  It is a little terse for the non-telepathic
perhaps, but contains the essential information (or most of it, anyway) on the
{\tt task} that executes the operation {\tt energy}.



Example 3: in-source documentation of routine {\tt sym\_symmetrize}

\begin{verbatim}
  subroutine sym_symmetrize(geom, basis, odensity, g_a)
  integer geom       ! [input] Geometry handle
  integer basis      ! [input] Basis handle
  integer g_a        ! [input] Global array to be symmetrized
  logical odensity   ! [input] true=density, false=hamiltonian
\end{verbatim}
Symmetrize a skeleton AO matrix (in global array with handle
\verb+g_a+) in the given basis set.  This is nothing more than
applying the projection operator for the totally symmetric
representation.
\begin{verbatim}
   B = (1/2h) * sum(R) [RT * (A + AT) * R]
\end{verbatim}
where \verb+R+ runs over all operators in the group (including
identity) and \verb+h+ is the order of the group.

Note that density matrices tranform according to slightly different
rules to Hamiltonian matrices if components of a shell (e.g.,
cartesian d's) are not orthonormal.  (see Dupuis and King, IJQC 11,
613-625, 1977).  Hence, specify \verb+odensity+ as \TRUE\ for
density-like matrices and \FALSE\ for all other totally symmetric
Hamiltonian-like operators.


{\em Comment:} This is about as good as it gets.

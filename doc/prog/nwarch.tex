\section{NWChem Architecture}
\label{sec:arch}

NWChem has a five-tiered modular architecture.  This structure is illustrated
conceptually by the diagram in Figure 1, which shows the five tiers and their
relationships to each other.
The first tier is the {\em generic task interface}.  This interface\footnote{Note that
this is an abstract programming interface, not a user interface.  The user's
'interface' with the code is the input file.} serves as the
mechanism that transfers control to the different modules in the second tier,
which consists of the {\em Molecular Calculation Modules}.
The molecular calculation modules are the high level programming
modules that accomplish computational tasks, performing particular operations
using the specified theories defined by the user input file.  These independent modules
of NWChem share data only through a disk-resident data base,
which allows modules to share data or to share access to files containing
data.
The third tier consists of the {\em Molecular
Modeling Tools}.  These routines provide basic chemical functionality such as symmetry,
basis sets, grids, geometry, and integrals.
The fourth tier is
the {\em Software
Development Toolkit}, which is the basic foundation of the code.
The fifth tier provides the {\em Utility Functions} needed by nearly all modules
in the code.  These include such functionality as input processing, output processing,
and timing.

In addition to using a modular approach for the design, NWChem is built 
on the concepts of object oriented programming (OOP) and non-uniform memory
access (NUMA).  The OOP approach might seem incompatible with a code written primarily
in Fortran77, since it does not
have all of the necessary functionality for an object oriented language (OOL).
However, many of the required features can be simulated by careful adherence
to the guidelines for encapsulation and data hiding outlined
in Section \ref{sec:coding-style}.
The main advantage of an object-oriented approach is that it 
allows for orderly and logical access
to data more-or-less independent of why or when a given module might require
the information.  In addition, it allows considerable flexibility in the 
manipulation and
distribution of data on shared memory, distributed memory, and massively
parallel hardware architectures, which is needed in a NUMA approach to parallel
computations.  However, this model does require that the program
developer have a fairly comprehensive understanding of the overall structure
of the code and the way in which the various parts fit together.

The following subsections describe this structure in broad outline, and refer
to specific chapters and sections in the code where
the various modules, tools, and "objects" are described in detail.


\subsection{Object Oriented Design}
\label{sec:ood}

The basic principles of object-oriented software development are abstraction, 
hierarchy, encapsulation, and modularity.  {\em Abstraction} is the separation of the
problem to be solved from the process used to solve it, which facilitates the
introduction of new methods as programming tools and hardware capabilities
evolve.  In complex systems, abstraction can be carried out on many levels,
resulting in a hierarchy that allows connections between many different components
and the development of further abstractions. {\em Encapsulation} is the creation
of isolated data structures or other objects in such a way that they can be
manipulated only in carefully controlled and well-defined ways, which helps to
reduce the problems due to unexpected interactions between components that are
supposed to be independent.  {\em Modularity}, which is the use of relatively
small program units having well-defined functionality, can also help reduce
interaction problems.  It can also aid overall code efficiency, if the modules
are written to be easily reused.
 
In an object oriented language such as C++, this methodology can be a feature
of the actual coding of the program.  NWChem is written in a mixture of C and
Fortran, however, and uses object oriented ideas at the design stage.
This requires some self-discipline on the part of the developers, but the
effort is well rewarded in improved implementation and easier code maintenance.
In a programming language such as Fortran77, which is not object oriented by
design, the concept of objects can be simulated  by developing a
well defined interface
for the programmer to use that in essence hides all of the gory details of 
"creating", "manipulating", and "destroying" an object.  The objects are
treated as if they
can be manipulated only through the interface.  In reality, of course, Fortran 77
allows the programmer to use any of the "private" data and routines
that are underneath the interface.  For this reason, the rules for encapsulation
and data hiding must be adhered to religiously, by following the guidelines outlined
in Section \ref{sec:coding-style}.

One of the basic features of an object is that all of the data and
functions related to the data are encapsulated and available only through
a "public" programming interface.  This encapsulation feature allows 
programmers to put related
data together in one object to be accessed in a well defined manner.
For example, the basis set object (described further in Section
\ref{sec:basis}) contains the number of basis functions, the
exponents, the coefficients and other data related to basis sets.
It also has a very well defined interface that can be used to access and
manipulate the data.  Because the data description, the internal "private"
functions, and the "public" interface together 
define the abstract concept of the object, specific examples of the
objects need to be created (instantiated).

Instantiations (or unique copies) of objects are simulated by allowing 
the user and the programmer to use different handles for different 
objects of the same type.
This feature gives the user the capability of defining
different basis sets during computation simply by naming different
basis
set objects (see Section \ref{sec:basis}).  For example, two different basis sets
can be defined for a molecule in an input file, as follows;

\begin{verbatim}
geometry
  Ne 0.0 0.0 0.0
end
basis "dz set"
  Ne library cc-pvdz
end
basis "qz set"
  Ne library cc-pvqz
end
set "ao basis" "dz set"
task scf
set "ao basis" "qz set"
task scf
task mp2
\end{verbatim}

The above example has two basis sets that have the same object abstraction,
(exponents and coefficients, etc.), but are different instantiations of
the object, \verb+"dz set"+ and \verb+"qz set"+, with different handles (i.e., names).
The handles can then be used to represent the currently "active" basis set for
the computation, using the input command
\verb+set "ao basis" "qz set"+.

Related to the object oriented design is the idea of an abstract programming
interface (API).  An API provides a common interface to many different
methods that perform the same type of task.  An API is different from an
object in the sense that there is no instantiation process.  Also, while the
functions are encapsulated, there is really no data that is encapsulated.
For example, memory objects, basis objects, and geometry objects are
passed into the integral API and integrals are passed back out in the
memory objects.  The integral API decides which of three different
integral packages will be used to compute the integrals.

\subsection{Non-Uniform Memory Access}

One of NWChem's design goals is to scale to massively parallel
hardware architectures in all aspects of the hardware: CPU, disk,
and memory.  With this goal in mind, distributing the data across
all of the nodes becomes necessary.
Therefore, in addition to the modular and object oriented architecture discussed
above, NWChem is built on the principle of non-uniform memory access (NUMA).
Just as a workstation has various levels of memory (registers, primary
and secondary cache, memory and swap space) with varying sizes and access
speed, distributing data across
nodes "simply" adds another level of remote memory.  The
programmer must be aware of this extra level of memory access when
designing the parallel algorithms in NWChem to get efficient, scalable
code.

The MA tool allows the programmer to allocate memory that is local to
the calling process.  This is data that will generally not be directly
shared with other processes, such as workspace for a particular local
calculation or for replication of very small sets of data.

The GA tool supports the NUMA model by allowing nodes to share arrays between
processes as if the memory is physically shared.  It allows the
programmer to use relatively simple routines to access and manipulate
data in the shared arrays.  However, the programmer must 
be aware that access to shared data will be slower than access
to local data.

Just as GA allows the programmer to effectively use the NUMA model for
memory, ChemIO is used to create files that are either local to the
process or
distributed among file systems.  This allows the programmer to perform
parallel I/O in the most efficient method for the particular
algorithm or the particular hardware.

Together, MA, GA, and ChemIO provide the tools needed to accomplish a
NUMA architecture.  They also form a significant part of the Software
Development Tooklkit layer.

\subsection{The Five-Tiered Modular Architecture}

With the basic understanding of the object oriented approach and the
NUMA approach, the programmer also needs to understand the basic
modular architecture that is used in NWChem.  This section provides a
basic overview of each of the tiers describes how 
they fit together to make a cohesive and extensible program.

\subsubsection{The Generic Task Interface}

In old-fashioned structured Fortran programming, the Generic Task
Interface would be refered to as the main program.  As the "interface"
between the user and the chemistry modules comprising NWChem, the
generic task interface processes the input, sets up the parallel 
environment, and performs any initialization needed for the
desired calculations.  It then transfers control to the appropriate
module, which performs the calculation.  After a particular task is
completed, control returns to the main program.
If the input specifies more than one task, 
control is transfered to the appropriate module for the next task.  This
process continues until all specified tasks have been completed, or
an error condition occurs.  When all tasks complete successfully, the
interface terminates program execution in an orderly manner.  When errors
occur, the interface tries to terminate program execution gracefully,
but the degree of success depends somewhat on the severity
of the error.  Chapter \ref{sec:generic} presents a detailed discussion
of the Generic Task Interface, and how it functions in NWChem.

\subsubsection{The Molecular Calculation Modules}

The second level of the five-tiered structure of NWChem consists of
the high level molecular calculation modules.  These are independent 
modules that perform the various functions of the code specified by the
task directives.  Examples 
include the self-consistent field (SCF) energy, the
SCF analytic gradient, and the density functional theory (DFT) energy 
modules.  The independent molecular calculation modules in NWChem 
can share data only through 
a run time database or through other well defined disk files.  
Each of the modules in this layer use
toolkits and routines in the lower layers of the architecture to
accomplish their tasks.  Chapter \ref{sec:modules} presents discussions
of each of the calculational modules in NWChem, and the various operations
that can be performed with these modules.

\subsubsection{The Molecular Modeling Toolkit}

The third level of the architecture of NWChem consists of
the molecular modeling toolkit.  Chapter \ref{sec:mmt} describes the
elements of this toolkit in detail,  including discussions of 
the geometry object (see Section \ref{sec:geometry}), 
the basis set object (see Section \ref{sec:basis}), the
linear algebra routines (see Section \ref{sec:la}), 
symmetry (see Section \ref{sec:sym}, and
the integral API (see Section
\ref{appendix_intapi}).  Each of these 
tools provides a basic functionality that is common to many of the
algorithms in chemistry.  The integral API provides a common interface to
the three integral packages available in NWChem.  The basis set object
provides the programmer with information related to a specific basis
set.  The geometry object provides the basic geometry in different
formats and provides for the definition of molecular as well as periodic
systems.  It also has information such as symmetry, atomic charges, and
atomic mass.  The linear algebra routines provide many general 
algorithms for basic vector-vector, vector-matrix, and matrix-matrix
operations, and for solving
eigenproblems.

\subsubsection{The Software Development Toolkit}

The Software Development Toolkit
makes up the foundation level of the five-tiered
structure of the code, and is the feature that makes it possible
to develop an object oriented code that is constructed mainly
in Fortran77.  Chapter \ref{sec:sdt} presents a detailed discussion
of this toolkit, which consists of four objects.
These are the runtime database (RTDB) (see Section
\ref{sec:rtdb}), the memory allocator (MA) (see Section \ref{sec:ma}),
Global Arrays (GA) (see Section \ref{sec:ga}), and ChemIO 
(see Section \ref{sec:ChemIO}).  Each of these tools provides the
interface between the chemistry specific part of the program and the
hardware.  They also support the NUMA parallel programming
paradigm used by NWChem.

The RTDB is a persistant data storage mechanism used in NWChem
to hold calculation specific information for the high level
programming modules.  Since NWChem does not destroy the RTDB at the end of
a calculation unless specifically directed to do so by the user, 
a given RTDB can be used in several independent
calculations.  The MA tool allocates memory that will be used local to
the processor.  The GA tool allocates memory that is distributed across
nodes (or shared in the case of shared memory machines) and is addressable
by all of the nodes.  ChemIO is a high performance I/O API 
designed to meet the requirements of large-scale computational
chemistry problems.  It allows the programmer to 
create I/O files that may be local or distributed.

\subsubsection{The Utility Routines}

This lowest level of the architecture contains many of the basic 
"odds-and-ends" functionality.
These are utility routines that most of the
above tiers need.
Examples include the timing routines,
the input parser, 
and the print routines.

